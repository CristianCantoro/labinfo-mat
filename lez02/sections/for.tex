\begin{frame}[fragile]\frametitle{Ciclo For (I)}
  Il ciclo \textbf{\texttt{for}} è un altro tipo di interazione dove si scorre una variabile,
  detta \textbf{indice} entro un intervallo di valori, con un dato \textbf{incremento}

  \begin{JavaCodePlain}[commandchars=\\!|]
    \Word!for| (\Blue!inzializzazione;| \Green!condizione;| \Violet!incremento|) {
      comandi;
    }
  \end{JavaCodePlain}

  Esempio:
  \begin{JavaCodePlain}[commandchars=\\!|]
    int s = 0;
    \Word!for| (\Blue!int n = 1;| \Green!n <= 100;| \Violet!n++|) {
      s = s + n;
    }
  \end{JavaCodePlain}

\end{frame}

\begin{frame}[fragile]\frametitle{Ciclo For (II)}
  Il ciclo \textbf{\texttt{for}} è un altro tipo di interazione dove si scorre una variabile,
  detta \textbf{indice} entro un intervallo di valori, con un dato \textbf{incremento}

  \begin{JavaCodePlain}[commandchars=\\!|]
    \Word!for| (\Blue!inzializzazione;| \Green!condizione;| \Violet!incremento|) {
      comandi;
    }
  \end{JavaCodePlain}

  Note:
  \begin{itemize}
   \item nell'\texttt{\Blue{inzializzazione;}} potete \textbf{dichiarare} una variabile;
   \item Per l'\texttt{\Violet{incremento;}} potete usare step anche diversi da 1;
  \end{itemize}

\end{frame}

\begin{frame}[fragile]\frametitle{Ciclo For (III)}
  Quando la condizione è \textbf{falsa} il ciclo viene interrotto:
  \begin{columns}[T]
    \begin{column}[T]{5cm}
      Pseudocodice:
      \begin{algorithmic}[1]     
	\State $s \gets 0$
	\For{$n \gets 1$ to $100$ by $1$}
	  \State $s \gets s + n$
	\EndFor
      \end{algorithmic}
    \end{column}
    \begin{column}[T]{6cm}
      Java:
      \begin{JavaCodePlain}[commandchars=\\!|]
	int s = 0;
	\Word!for| (\Blue!int n = 1;| \Green!n <= 100;| \Violet!n++|) {
	  s = s + n;
	}
      \end{JavaCodePlain}
    \end{column}
  \end{columns}
\end{frame}

\begin{frame}{Ciclo For (IV)}

  % Define block styles
  \tikzstyle{decision} = [diamond, draw, fill=green!20,text width=3.5em, text
                          badly centered, node distance=1.5cm, inner sep=0pt, font=\scriptsize]
  \tikzstyle{block} = [rectangle, draw, rounded corners]
  \tikzstyle{line} = [draw, -latex']

  Diagrama di flusso dell'istruzione for:
  \begin{center}
    \begin{tikzpicture}[node distance = 1cm, auto]
	% Place nodes
	\node [block, fill=blue!20!white]                                   (init)      {\texttt{int s = 0;}};
	\node [block, below of=init, fill=blue!5!white, node distance=1cm] (forinit)   {\texttt{int n = 1;}};
	\node [decision, below of=forinit]                                  (decide)    {$n \leq 100$};
	\node [block, fill=red!50, below of=decide, node distance=1.5cm]    (comandi)   {\texttt{comandi;}};
	\node [block, fill=violet!30!white, below of=comandi]               (incremento){\texttt{i++}};
	\node [circle, below of=incremento]         (vuoto) {};
	\node [block, right of=vuoto, node distance=3cm]                    (stop) {resto del programma};
	% Draw edges
	\path [line] (init)       -- (forinit);
	\path [line] (forinit)    -- (decide);
	\path [line] (decide)     --  node {sì}(comandi);
	\path [line] (comandi)    -- (incremento);
	\path [line] (decide)  --++ (3,0) node[anchor=south,pos=0.5]{no} -| (stop);
	\path [line] (incremento)  --++ (0,-1) --++ (-2,0)  |- (decide);
    \end{tikzpicture}   
  \end{center}

\end{frame}

\begin{frame}[fragile]\frametitle{Esempi di ciclo \texttt{for}}

  \begin{itemize} 
    \item Nel ciclo $i$ assume i valori $i = 0, 5, 10, 15, \dots$
  \end{itemize}
  \begin{JavaCodePlain}[commandchars=\\!|]
    \Word!for| (\Blue!int i = 0;| \Green!i < 100;| \Violet!i = i + 5|) {
      comandi;
    }
  \end{JavaCodePlain}
  \begin{itemize}
    \item[$\Rightarrow$] Al termine del ciclo $i$ vale $100$.
  \end{itemize}

  \begin{itemize} 
    \item Nel ciclo $j$ assume i valori $j = 1, 2, 4, \dots$
  \end{itemize}
  \begin{JavaCodePlain}[commandchars=\\!|]
    \Word!for| (\Blue!int j = 1;| \Green!j < 1024;| \Violet!j = j * 2|) {
      comandi;
    }
  \end{JavaCodePlain}
  \begin{itemize}
   \item[$\Rightarrow$] Al termine del ciclo $j$ vale $1024$.
  \end{itemize}
  
  \begin{itemize} 
    \item Nel ciclo $k$ assume i valori $k = 10, 9, 8, \dots$
  \end{itemize}    
  \begin{JavaCodePlain}[commandchars=\\!|]
    \Word!for| (\Blue!int k = 10;| \Green!k > 0;| \Violet!k--|) {
      comandi;
    }
  \end{JavaCodePlain}
  \begin{itemize}
   \item[$\Rightarrow$] Al termine del ciclo $k$ vale $0$.
  \end{itemize}

  Esistono gli operatori \texttt{+=}, \texttt{-=}, \texttt{*=}, ma vi consiglio di scrivere
  l'incremento in modo espicito per evitare errori.

\end{frame}

\begin{frame}[fragile]\frametitle{Analogie tra ciclo \texttt{for} e ciclo \texttt{while}}

  I cicli \textbf{\texttt{for}} e \textbf{\texttt{while}} sono equivalenti. Tutti i cicli \textbf{\texttt{for}} 
  possono essere ``tradotti'' in cicli \textbf{\texttt{while}} e \textbf{viceversa}.
  
  \begin{JavaCodePlain}[commandchars=\\!|]
    \Word!for| (\Blue!inzializzazione;| \Green!condizione;| \Violet!incremento|) {
      comandi;
    }
  \end{JavaCodePlain}

  \begin{JavaCodePlain}[commandchars=\\!|]
    \Blue!inzializzazione;|
    \Word!while| (\Green!condizione|) {
      comandi;
      \Violet!incremento;|
    }
  \end{JavaCodePlain}
\end{frame}
