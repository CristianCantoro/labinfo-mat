\begin{frame}[fragile]\frametitle{Istruzione If-Else}

  \begin{JavaCodePlain}[commandchars=\\!|]
    int n = 4;
    if (n \% 2 == 0) {
      System.out.println(\String!"Il numero è pari"|);
    } else {
      System.out.println(\String!"Il numero è dispari"|);
    }
  \end{JavaCodePlain}

\end{frame}


% \begin{frame}[fragile]\frametitle{Istruzione If-Else}
% 
%   \begin{verbatim}
%     int n = 4;
%   \end{verbatim}
%   {\color{green!50!black}
%     \begin{verbatim}
%      if (n % 2 == 0)
%     \end{verbatim}
%   }{\color{red}
%     \begin{verbatim}
%       {
%         System.out.println("Il numero è pari");}
%       }
%     \end{verbatim}
%   }
%   \begin{verbatim}
%     else
%   \end{verbatim}
%   {\color{red}
%     \begin{verbatim}
%       {
%         System.out.println("Il numero è dispari");}
%       }
%     \end{verbatim}
%   }
% \end{frame}


\begin{frame}[fragile]\frametitle{Istruzione If-Else}
  \begin{JavaCodePlain}[commandchars=\\!|]
    \Blue!int n = 4;|
    \Green!if (n \% 2 == 0)| {
      \Red!System.out.println("Il numero è pari");|
    } else {
      \Red!System.out.println("Il numero è dispari");|
    }
  \end{JavaCodePlain}
\end{frame}

\begin{frame}{Istruzione If-Then-Else}

  % Define block styles
  \tikzstyle{decision} = [diamond, draw, fill=green!20,text width=4.5em, text
                          badly centered, node distance=3cm, inner sep=0pt, font=\scriptsize]
  \tikzstyle{block} = [rectangle, draw, rounded corners]
  \tikzstyle{line} = [draw, -latex']

  Diagramma di flusso dell'istruzione If-Then-Else:
  \begin{center}
    \begin{tikzpicture}[node distance = 1cm, auto]
	% Place nodes
	\node [block, text width=1.75cm, fill=blue!20!white] (init) {\texttt{int n = 1;}\\\texttt{int s = 0;}};
	\node [decision, below of=init] (decide) {\textbf{\texttt{if}} $n \; \mathit{mod} \; 2 = 0$};
	\node [block, left  of=decide, node distance=4cm, fill=red!50, text width=3cm, text centered] (comandi_else){Il numero è dispari};
	\node [block, right of=decide, node distance=4cm, fill=red!50, text width=3cm, text centered] (comandi_if)  {Il numero è pari};
	\node [block, below of=decide, node distance=2cm] (stop) {resto del programma};
	% Draw edges
	\path [line] (init) -- (decide);
	\path [line] (decide) -- node {no}(comandi_else);
	\path [line] (decide) -- node {sì}(comandi_if);
	\path [line] (comandi_if) --++ (0,-2) -- (stop);
	\path [line] (comandi_else) --++ (0,-2) -- (stop);
    \end{tikzpicture}   
  \end{center}

\end{frame}

