\begin{frame}{Iterazione (I)}
  Definzione di iterazione\footnote{\url{https://it.wikipedia.org/wiki/Iterazione\#Informatica}}:
  \begin{quote}
    «L'\textbf{iterazione}, chiamata anche \textbf{ciclo} o con il termine inglese \textbf{loop}, 
    è una struttura di controllo [...]  che ordina all'elaboratore di eseguire ripetutamente 
    una sequenza di istruzioni, solitamente fino al verificarsi di particolari condizioni logiche specificate.»
  \end{quote}
\end{frame}

\begin{frame}{Iterazione (II)}
  L'iterazione è utile per ``tradurre'' sommatorie e prodotturie:
  \begin{itemize}
   \item \[S(100) = \sum_{n=1}^{n=100} n\]
	 calcola la somma di $n$ \textbf{per $n$ che va da  $1$ a $100$}
	 (calcola la somma dei primi $100$ interi)
   \item \[n! = \prod_{i=1}^{i=n} i\]
	 calcola il prodotto di $i$ \textbf{per $i$ che va da  $1$ a $n$}
	 (il fattoriale di $n$ \`e il prodotto dei numeri da $1$ a $n$.)
  \end{itemize}

\end{frame}

\begin{frame}{Iterazione (III)}
  L'iterazione traduce naturalmente formule ricorsive:
  \begin{itemize}
    \item Metodo della bisezione: metodo che trova numericamente gli zeri di una funzione continua;

    \item Metodo per il calcolo della radice quadrata di Newton.
    Si può calcolare la radice quadrata di un numero $z$ nel modo seguente:
    \begin{equation*}
      \left\{\begin{aligned}
	  & x_0 = 0.5\\
	  & x_{n +1} = 0.5 \cdot {x_n} (3  - zx^2_n)
      \end{aligned}\right.
    \end{equation*}
    \begin{equation*}
     \lim_{n \to \infty} x_{n} = \sqrt{z}
    \end{equation*}
    Posso calcolare l'errore compiuto al termine $n$-esimo, $x_{n}$ come $\varepsilon_{n} = |x^{2}_{n} - z|$,
    Posso calcolare $\sqrt{z}$ a una data precisione: ``prosegui a calcolare nuovi termini della successione $x_{n}$
    \textbf{finché l'errore non è più piccolo di $10^{-3}$}''.


  \end{itemize}

\end{frame}

\begin{frame}[fragile]\frametitle{Ciclo While (I)}
  \textbf{Finché} la condizione è \textbf{vera} esegui una certa operazione (ciclo \textbf{\texttt{while-do}}
  o semplicemente \textbf{\texttt{while}}):
  \begin{columns}[T]
    \begin{column}[T]{5cm}
      Pseudocodice:
      \begin{algorithmic}[1]     
	\State $s \gets 0$
	\State $n \gets 1$
	\While{$n \leq 100$}
	  \State $s \gets s + n$
	  \State $n \gets n + 1$
	\EndWhile
      \end{algorithmic}
    \end{column}
    \begin{column}[T]{5cm}
      Java:
      \begin{JavaCodePlain}[commandchars=\\!|]
	int s = 0;
	int n = 1;
	\Word!while| (\Green!n <= 100|) {
	  \Red!s = s + n;|
	  \Red!n = n + 1;|
	}
      \end{JavaCodePlain}
    \end{column}
  \end{columns}
  ${}$\\
  Nel momento in cui la condizione all'interno del while diventa falsa si \textbf{esce} dal ciclo.
\end{frame}

\begin{frame}[fragile]\frametitle{Ciclo While (II)}
  \textbf{Finché} la condizione è \textbf{vera} esegui una certa operazione (ciclo \textbf{\texttt{while-do}}):
  \begin{columns}[T]
    \begin{column}[T]{5cm}
      Pseudocodice:
      \begin{algorithmic}[1]     
	\While{condizione}
	  \State comandi
	\EndWhile
      \end{algorithmic}
    \end{column}
    \begin{column}[T]{5cm}
      Java:
      \begin{JavaCodePlain}[commandchars=\\!|]
	\Word!while| (\Green!condizione|) {
	  \Red!comandi;|
	}
      \end{JavaCodePlain}
    \end{column}
  \end{columns}
  ${}$\\
  Nel momento in cui la condizione all'interno del while diventa falsa si \textbf{esce} dal ciclo.
\end{frame}

\begin{frame}{Ciclo While (III)}

  % Define block styles
  \tikzstyle{decision} = [diamond, draw, fill=green!20,text width=4.5em, text
                          badly centered, node distance=3cm, inner sep=0pt]
  \tikzstyle{block} = [rectangle, draw, rounded corners]
  \tikzstyle{line} = [draw, -latex']

  Diagrama di flusso dell'istruzione While:
  \begin{center}
    \begin{tikzpicture}[node distance = 1cm, auto]
	% Place nodes
	\node [block, text width=1.75cm, fill=blue!20!white] (init) {\texttt{int n = 1;}\\\texttt{int s = 0;}};
	\node [draw,circle,minimum size=1mm,inner sep=0pt,outer sep=0pt,fill=black, below of=init] (vuoto) {};
	\node [decision, below of=init] (decide) {$n \leq 100$};
	\node [block, below of=decide, node distance=2cm] (stop) {resto del programma};
	\node [block, right of=decide, node distance=4cm, fill=red!50, text width=1.75cm] (comandi) {\texttt{s = s + n;}\\ \texttt{n = n + 1;}};
	% Draw edges
	\path [line] (init) -- (decide);
	\path [draw] (comandi)  --++ (0,2) |- (vuoto);
	\path [line] (decide) -- node {sì}(comandi);
	\path [line] (decide) -- node {no}(stop);
    \end{tikzpicture}   
  \end{center}

\end{frame}

\begin{frame}[fragile]\frametitle{Ciclo While (IV)}

  Esecuzione del ciclo passo-passo:
  \begin{table}[]
  \centering
    \begin{tabular}{|c|c|c|c|}
    \hline
    \textbf{step} & \textbf{$n$} & \textbf{$s$} & \textbf{$n \leq 100 $?} \\ \hline
      $-$  & $1$   & $0$    & $-$              \\ \hline
      1    & $1$   & $0$    & sì               \\ \hline
      2    & $2$   & $1$    & sì               \\ \hline
      3    & $3$   & $3$    & sì               \\ \hline
      4    & $4$   & $6$    & sì               \\ \hline
      ...  & ...   & ...    & ...              \\ \hline
      99   & $99$  & $4851$ & sì               \\ \hline
      100  & $100$ & $4950$ & sì               \\ \hline
      101  & $101$ & $5050$ & no               \\ \hline
% %       $-$  & $101$ & $5050$ &                  \\  \hline   
    \end{tabular}
  \end{table}

\end{frame}

\begin{frame}[fragile]\frametitle{Ciclo While (VI)}
  Attenzione agli estremi dei cicli:
  \begin{columns}[T]
    \begin{column}[T]{5cm}
      \begin{JavaCodePlain}[commandchars=\\!|]
	int n = 1;
	\Word!while| (\Green!n <= 10|) {
	  \Word!println|(\String!"Ciao, mondo\bang"|); 
	  n = n + 1; 
	}
      \end{JavaCodePlain}
      Scrive \texttt{\String{Ciao, mondo!}} \textbf{10} volte.
    \end{column}
    \begin{column}[T]{5cm}
      \begin{JavaCodePlain}[commandchars=\\!|]
	int n \Red!= 0|;
	\Word!while| (\Green!n \Red!< 10||) {
	  \Word!println|(\String!"Ciao, mondo\bang"|); 
	  n = n + 1; 
	}
      \end{JavaCodePlain}
      Scrive \texttt{\String{Ciao, mondo!}} \textbf{10} volte.
    \end{column}
  \end{columns}
\end{frame}

\begin{frame}[fragile]\frametitle{Ciclo While (VI)}
  Se la condizione non diventa mai falsa, allora il ciclo \textbf{non termina mai}.
  \begin{columns}[T]
    \begin{column}[T]{5cm}
      \begin{JavaCodePlain}[commandchars=\\!|]
	int s = 0;
	int n = 99;
	\Word!while| (\Green!n \bang= 0|) {
	  s = s + n; 
	  \Green!// n = 99, \dots, 1,\Red!-1, \dots||
	  n = n - 2; 
	}
      \end{JavaCodePlain}
    \end{column}
    \begin{column}[T]{5cm}
      \begin{JavaCodePlain}[commandchars=\\!|]
	\Red!int p = 0;|
	int n = 1;
	\Word!while| (\Green!p < 100|) {
	  p = p * n; \Green!// p == 0|
	  n = n + 1;
	}
      \end{JavaCodePlain}
    \end{column}
  \end{columns}
\end{frame}

\begin{frame}[fragile]\frametitle{Ciclo While (VII)}
  \begin{columns}[T]
    \begin{column}[T]{5cm}
      Se la condizione è \textbf{sempre vera}, allora il ciclo non termina mai:
      \begin{JavaCodePlain}[commandchars=\\!|]
	\Word!while| (\Green!\textbf!true||) {
	  s = s + n; 
	  n = n + 1; 
	}
      \end{JavaCodePlain}
      Il comando $\Word{\textbf{break}}$ permette di uscire dal ciclo.
      \begin{JavaCodePlain}[commandchars=\\!|]
	int s = 0;
	int n = 1;
	\Word!while| (\Green!n <= 50|) {
	  \Green!// se n = 33 esco|
	  \Word!if| (n == 33) {
	    \Word!break|;
	  }
	  s = s + n;
	  n = n + 1;
	}
      \end{JavaCodePlain}

    \end{column}
    \begin{column}[T]{5cm}
      Se devo saltare dei valori allora posso usare il comando $\Word{\textbf{continue}}$:
      \[ s = \sum_{i=1, i \neq 3}^{50} i \]
      \begin{JavaCodePlain}[commandchars=\\!|]
	int s = 0;
	int n = 1;
	\Word!while| (\Green!n <= 50|) {
	  \Green!// salto il caso n = 3|
	  \Word!if| (n == 3) {
	    n = n + 1;
	    \Word!continue|;
	  }
	  s = s + n;
	  n = n + 1;
	}
      \end{JavaCodePlain}
    \end{column}
  \end{columns}
\end{frame}

\begin{frame}{Nuovo progetto Eclipse}
  \begin{enumerate}
   \item Aprire Eclipse
   \item \emph{File} $\rightarrow$ \emph{New} $\rightarrow$ \emph{Project} $\rightarrow$ \emph{Java Project}
   \item Inserire il nome e click su \emph{Finish}
   \item Click destro su \emph{src} $\rightarrow$ \emph{New} $\rightarrow$ \emph{Class\dots}
   \item Inserire il nome e check su \newline
	  \texttt{public static void main(String[] args)}
   \item Click su \emph{Finish}
  \end{enumerate}
  Se avete difficoltà o problemi scrivetemi una mail!

\end{frame}

\begin{frame}{Esercizi (I)}
  \begin{enumerate}
   \item Scrivete un programma che stampi la stringa \texttt{\String{Ciao, mondo!}} a schermo per 10 volte;
   \item Scrivere un programma che stampi tutti i numeri pari fino a 1000;
  \end{enumerate}
\end{frame}

\begin{frame}[fragile]\frametitle{Ciclo Do-While (I)}
  Esegui una certa operazione \textbf{finché} la condizione è \textbf{vera} (ciclo \textbf{\texttt{do-while}}):
  \begin{columns}[T]
    \begin{column}[T]{5cm}
      Pseudocodice:
      \begin{algorithmic}[1]     
	\State $num \gets 20$
	\State $count \gets 0$
	\Do
	  \State $num \gets num / 2$
	  \State $count \gets count + 1$
	\DoWhile{$num \; \mathit{mod} \; 2 \neq 0$}
      \end{algorithmic}
    \end{column}
    \begin{column}[T]{5cm}
      Java:
      \begin{JavaCodePlain}[commandchars=\\!|]
	int num = 20;
	int count = 0;
	\Word!do| {
	  num = num / 2;
	  count = count + 1;
	} \Word!while| (\Green!num \% 2 \bang= 0|)
      \end{JavaCodePlain}
    \end{column}
  \end{columns}
  ${}$\newline
  La differenza fondamentale tra la forma \texttt{while} e quella \texttt{do-while} 
  è che con il \texttt{do-while} i comandi all'interno del ciclo vengono 
  \textbf{sempre} eseguiti \textbf{almeno una volta}.
\end{frame}

\begin{frame}{Ciclo Do-While (II)}

  % Define block styles
  \tikzstyle{decision} = [diamond, draw, fill=green!20,text width=5.5em, text
                          badly centered, node distance=2cm, inner sep=0pt, font=\scriptsize]
  \tikzstyle{block} = [rectangle, draw, rounded corners]
  \tikzstyle{line} = [draw, -latex']

  Diagrama di flusso dell'istruzione Do-While:
  \begin{center}
    \begin{tikzpicture}[node distance = 1cm, auto]
	% Place nodes
	\node [block, text width=1.75cm, fill=blue!20!white, text width=3.5cm] (init) {\texttt{int num = 20;}\\\texttt{int count = 0;}};
	\node [block, below of=init, fill=red!50, text width=3.5cm, node distance=1.5cm] (comandi) {\texttt{num = num / 2;}\\ \texttt{count = count + 1;}};
	% \node [draw,circle,minimum size=1mm,inner sep=0pt,outer sep=0pt,fill=black, below of=init] (vuoto) {};
	\node [decision, below of=comandi] (decide) {\texttt{\textbf{while}}\\$num \; \mathit{mod} \; 2 \neq 0$};
	\node [block, below of=decide, node distance=2cm] (stop) {resto del programma};
	% Draw edges
	\path [line] (init) -- (comandi);
	\path [line] (comandi) -- (decide);
	\path [line] (decide) -- node {no}(stop);
	\path [line] (decide)  --++ (3,0) node[anchor=south,pos=0.5]{sì} |- (comandi);
    \end{tikzpicture}   
  \end{center}

\end{frame}
