\documentclass[10pt]{beamer}

\mode<presentation>
{
  \usetheme{Berlin}
  \setbeamertemplate{blocks}[rounded]
  \setbeamercolor{block title}{bg=gray}
  \setbeamercovered{transparent}
}

% %%% Packages %%%
% Babel and fonts
\usepackage[english]{babel}
\usepackage[utf8]{inputenc}
\usepackage{times}

% Graphics and images
\usepackage{graphicx}
\DeclareGraphicsExtensions{.pdf,.png,.jpg, .eps}
\usepackage{relsize}
\usepackage{colortbl}


% Math notation and symbols
\usepackage{mathrsfs}
\usepackage{amsthm}
\usepackage{bbold}
\usepackage{amsfonts}
\usepackage{amsmath}
\usepackage{amssymb}
% \usepackage{algorithm}
\usepackage{algpseudocode}
\usepackage{listings}

\DeclareMathOperator*{\len}{\textbf{length of }}

% \DeclareMathOperator*{\len}{\text{length of }}
% Backup slides
\usepackage{appendixnumberbeamer}

% Tables, Colums and the like
\usepackage{longtable}
\usepackage{listings}

% Hyperref
\usepackage{hyperref} 

% Boxes
\usepackage{fancybox}
\usepackage{lmodern}
\usepackage{tikz}
\usepackage{tcolorbox}
\usepackage{mdframed}	

% Custom packages
\usepackage{presmacros}

\usepackage{hyperref}
\usepackage{listings}
\usepackage{setspace}
\usepackage{subfiles}
\usepackage{alltt}
\usepackage{mathtools}
\usepackage{fancyvrb}
\usepackage{url}

\usepackage{tikz}
\usetikzlibrary{shapes,arrows}

\lstdefinestyle{JavaPlain}{ %
basicstyle=\scriptsize\ttfamily, % the size of the fonts 
numbers=left,                   % where to put the line-numbers
numberstyle=\tiny,      % the size of the fonts that are used for th
stepnumber=1,                   % the step between two line-numbers
numbersep=5pt,                  % how far the line-numbers are from the code
backgroundcolor=\color{white},  % choose the background color
showspaces=false,               % show spaces adding particular underscores
showstringspaces=false,         % underline spaces within strings
showtabs=false,                 % show tabs within strings adding 
frame=single,           % adds a frame around the code
tabsize=2,          % sets default tabsize to 2 spaces
captionpos=b,           % sets the caption-position to bottom
breaklines=true,        % sets automatic line breaking
breakatwhitespace=false,    % sets if automatic breaks should only happen
fancyvrb=true,
fvcmdparams=textbf 1 textit 1,
}

\newcommand\Red[1]{\textcolor{red}{#1}}
\newcommand\Green[1]{\textcolor{green!50!black}{#1}}
\newcommand\Blue[1]{\textcolor{blue!60!white}{#1}}
\newcommand\Violet[1]{\textcolor{violet}{#1}}
\newcommand\String[1]{\textcolor{blue!80!blue}{#1}}
\newcommand\Word[1]{\textcolor{purple!90!red}{#1}}
\newcommand\bang{!}

\newenvironment{JavaCodePlain}[1][]
  { \VerbatimEnvironment%
    \begin{Verbatim}[#1]}
  { \end{Verbatim}  } 

\algdef{SE}[DOWHILE]{Do}{DoWhile}{\algorithmicdo}[1]{\algorithmicwhile\ #1}%
\renewcommand{\algorithmicrequire}{\textbf{Input:}}
\renewcommand{\algorithmicensure}{\textbf{Output:}}

\AtBeginSection[]
{
  \begin{frame}<beamer>
    \frametitle{Outline for section \thesection}
    \tableofcontents[currentsection]
  \end{frame}
}

\begin{document}

% \pgfdeclareimage[height=0.5 cm]{dblogo-white}{img/logo-white.png}
\setbeamertemplate{navigation symbols}{}

% \begin{frame}[plain]
%   \pgfdeclareimage[height=\paperheight]{mountain}{img/mountain.png}
%   \begin{tikzpicture}[remember picture,overlay]
%       \node[at=(current page.center)] {
% 	\pgfuseimage{mountain}
%       };
%   \end{tikzpicture}
% 
% 
%   \begin{center}
%     \usebeamerfont{title}\textcolor{white}{\inserttitle}\par
%   \end{center}
% 
% %   \begin{flushleft}
% %     \begin{small}  
% %       \textcolor{white}{presented by Cristian Consonni}
% %     \end{small}
% %   \end{flushleft}
% 
%   \begin{flushright}   
%     \pgfuseimage{dblogo-white}
%   \end{flushright}
% 
% \end{frame}

\begin{frame}
  \titlepage
\end{frame}

\begin{frame}{Outline}
  \tableofcontents
\end{frame}


\subsection[Informazioni generali]{Informazioni generali}

\begin{frame}{Chi sono}
  Cristian Consonni

  \begin{itemize}
    \item \textbf{DISI - Dipartimento di Ingegneria e Scienza dell'Informazione}
    \item \textbf{Pagina web} del laboratorio: \structure{\url{http://disi.unitn.it/~consonni/teaching}}
    \item \textbf{Email}: \structure{\url{cristian.consonni@unitn.it}}
    \item \textbf{Ufficio}: Povo 2 - Open Space 9
      \begin{itemize}
	\item Per domande: scrivetemi una mail
	\item Ricevimento: su appuntamento via mail
      \end{itemize}
  \end{itemize}
\end{frame}



\begin{frame}{Obiettivi del laboratorio}
  Obiettivi del laboratorio:
  \begin{itemize}
    \item Apprendere i fondamenti di un vero linguaggio di programmazione (Java)
    \item Svolgere il progetto
  \end{itemize}

  Obiettivi del laboratorio
  \begin{enumerate}
    \item Fare esperienza in laboratorio
    \item Raggiungere una buona manualità nell'uso degli strumenti standard
    \item Esercizi
  \end{enumerate}

\end{frame}

\pgfdeclareimage[width=0.6\paperwidth]{xkcd}{img/11th_grade.png}
\begin{frame}{Manualità (I)}
  \begin{center}
    \pgfuseimage{xkcd}
  \end{center}
  \url{https://xkcd.com/519/}
\end{frame}

\pgfdeclareimage[width=0.6\paperwidth]{abstrusegoose}{img/ars_longa_vita_brevis.png}
\begin{frame}{Manualità (II)}
  \textbf{How to Teach Yourself Programming:}\footnote{\url{http://abstrusegoose.com/249}}
  \begin{center}
    \pgfuseimage{abstrusegoose}
  \end{center}
  
\end{frame}

\begin{frame}{Slides}

  Info sulle slide:
  \begin{itemize}
    \item le slide del corso saranno rese disponibili sul sito;
    \item segnalate pure eventuali errori;
    \item cercherò di pubblicare le slide in anticipo rispetto alla lezione;
    \item queste slide sono prodotte con \LaTeX \; \texttt{Beamer} (\emph{usate} \LaTeX!);
  \end{itemize}

  Segnalazioni di materiale:
  \begin{itemize}
    \item Materiale da voi prodotto;
    \item Cose interessanti che trovate online;
    \item Possiamo valutare insieme se riutilizzarle;
  \end{itemize}
\end{frame}

\section{Commenti e Stampa a schermo}
\begin{frame}[fragile]\frametitle{Commenti}

  \begin{itemize}
   \item Commento su singola riga: doppio slash $//$
  \end{itemize}

  \begin{center}
    \begin{minipage}[c]{8cm}
      \begin{JavaCodePlain}[commandchars=\\!|]
	\Green!// Commento su singola riga|
	\end{JavaCodePlain}
    \end{minipage}  
  \end{center}

  \begin{itemize}
   \item Commento su singola riga: (apertura) slash + asterisco $/*$ (chiusura) asterisco + slash $*/$
  \end{itemize}

  \begin{center}
    \begin{minipage}[c]{8cm}
      {\color{green!50!black}
	\begin{verbatim}
	  /*
	  Commento multi riga.
	  Questo commento si può dipanare
	  su più di una riga
	  */
	\end{verbatim}
      }
    \end{minipage}  
  \end{center}
\end{frame}

\begin{frame}[fragile]\frametitle{Commenti (II)}

  Usate sempre commenti significativi:
  \begin{itemize}
   \item {\color{green} OK}:
  \end{itemize}
  \begin{center}
    \begin{minipage}[c]{11cm}
      {\color{green!50!black}
	\begin{verbatim}
	  // inizializzo max a un valore che è sicuramente
	  // più piccolo dei valori assunti dal vettore
	  int max = -1;
	\end{verbatim} 
      }
    \end{minipage}  
  \end{center}

  \begin{itemize}
   \item {\color{red} NO}:
  \end{itemize}
  {\color{red!80!black}
  \begin{center}
    \begin{minipage}[c]{11cm}
      \begin{verbatim}
	// inizializzo max a -1
	int max = -1;
      \end{verbatim}
    \end{minipage}  
  \end{center}
  }

\end{frame}

\begin{frame}{Stampa di stringe a schermo}
  \begin{itemize}
    \item Concatenazione di stringhe: \newline
	  \texttt{System.out.println(\String{"Ciao, "} + name + \String{"!"})}
    \item Print formatted: \newline
	  \texttt{System.out.printf(\String{"Ciao, \%s!"}, name)}
      \begin{itemize}
	\item \texttt{\%d} per stampare \emph{interi} (\texttt{int});
	\item \texttt{\%f} per stampare \emph{numeri con la virgola} (\texttt{float}, \texttt{double});
	\item \texttt{\%s} per stampare \emph{stringhe} (\texttt{String});
      \end{itemize}
  \end{itemize}
 
\end{frame}


\section{Strutture di controllo}
\begin{frame}[fragile]\frametitle{Istruzione If-Else}

  \begin{JavaCodePlain}[commandchars=\\!|]
    int n = 4;
    if (n \% 2 == 0) {
      System.out.println(\String!"Il numero è pari"|);
    } else {
      System.out.println(\String!"Il numero è dispari"|);
    }
  \end{JavaCodePlain}

\end{frame}


% \begin{frame}[fragile]\frametitle{Istruzione If-Else}
% 
%   \begin{verbatim}
%     int n = 4;
%   \end{verbatim}
%   {\color{green!50!black}
%     \begin{verbatim}
%      if (n % 2 == 0)
%     \end{verbatim}
%   }{\color{red}
%     \begin{verbatim}
%       {
%         System.out.println("Il numero è pari");}
%       }
%     \end{verbatim}
%   }
%   \begin{verbatim}
%     else
%   \end{verbatim}
%   {\color{red}
%     \begin{verbatim}
%       {
%         System.out.println("Il numero è dispari");}
%       }
%     \end{verbatim}
%   }
% \end{frame}


\begin{frame}[fragile]\frametitle{Istruzione If-Else}
  \begin{JavaCodePlain}[commandchars=\\!|]
    \Blue!int n = 4;|
    \Green!if (n \% 2 == 0)| {
      \Red!System.out.println("Il numero è pari");|
    } else {
      \Red!System.out.println("Il numero è dispari");|
    }
  \end{JavaCodePlain}
\end{frame}

\begin{frame}{Istruzione If-Then-Else}

  % Define block styles
  \tikzstyle{decision} = [diamond, draw, fill=green!20,text width=4.5em, text
                          badly centered, node distance=3cm, inner sep=0pt, font=\scriptsize]
  \tikzstyle{block} = [rectangle, draw, rounded corners]
  \tikzstyle{line} = [draw, -latex']

  Diagramma di flusso dell'istruzione If-Then-Else:
  \begin{center}
    \begin{tikzpicture}[node distance = 1cm, auto]
	% Place nodes
	\node [block, text width=1.75cm, fill=blue!20!white] (init) {\texttt{int n = 1;}\\\texttt{int s = 0;}};
	\node [decision, below of=init] (decide) {\textbf{\texttt{if}} $n \; \mathit{mod} \; 2 = 0$};
	\node [block, left  of=decide, node distance=4cm, fill=red!50, text width=3cm, text centered] (comandi_else){Il numero è dispari};
	\node [block, right of=decide, node distance=4cm, fill=red!50, text width=3cm, text centered] (comandi_if)  {Il numero è pari};
	\node [block, below of=decide, node distance=2cm] (stop) {resto del programma};
	% Draw edges
	\path [line] (init) -- (decide);
	\path [line] (decide) -- node {no}(comandi_else);
	\path [line] (decide) -- node {sì}(comandi_if);
	\path [line] (comandi_if) --++ (0,-2) -- (stop);
	\path [line] (comandi_else) --++ (0,-2) -- (stop);
    \end{tikzpicture}   
  \end{center}

\end{frame}



\section{Ciclo While}
\begin{frame}{Iterazione (I)}
  Definzione di iterazione\footnote{\url{https://it.wikipedia.org/wiki/Iterazione\#Informatica}}:
  \begin{quote}
    «L'\textbf{iterazione}, chiamata anche \textbf{ciclo} o con il termine inglese \textbf{loop}, 
    è una struttura di controllo [...]  che ordina all'elaboratore di eseguire ripetutamente 
    una sequenza di istruzioni, solitamente fino al verificarsi di particolari condizioni logiche specificate.»
  \end{quote}
\end{frame}

\begin{frame}{Iterazione (II)}
  L'iterazione è utile per ``tradurre'' sommatorie e prodotturie:
  \begin{itemize}
   \item \[S(100) = \sum_{n=1}^{n=100} n\]
	 calcola la somma di $n$ \textbf{per $n$ che va da  $1$ a $100$}
	 (calcola la somma dei primi $100$ interi)
   \item \[n! = \prod_{i=1}^{i=n} i\]
	 calcola il prodotto di $i$ \textbf{per $i$ che va da  $1$ a $n$}
	 (il fattoriale di $n$ \`e il prodotto dei numeri da $1$ a $n$.)
  \end{itemize}

\end{frame}

\begin{frame}{Iterazione (III)}
  L'iterazione traduce naturalmente formule ricorsive:
  \begin{itemize}
    \item Metodo della bisezione: metodo che trova numericamente gli zeri di una funzione continua;

    \item Metodo per il calcolo della radice quadrata di Newton.
    Si può calcolare la radice quadrata di un numero $z$ nel modo seguente:
    \begin{equation*}
      \left\{\begin{aligned}
	  & x_0 = \dfrac{1}{\lfloor x \rfloor} \\
	  & x_{n +1} = 0.5 \cdot {x_n} (3  - zx^2_n)
      \end{aligned}\right.
    \end{equation*}
    \begin{equation*}
     \lim_{n \to \infty} x_{n} = \dfrac{1}{\sqrt{z}}
    \end{equation*}
    Posso calcolare l'errore compiuto al termine $n$-esimo, $x_{n}$ come 
    $\varepsilon_{n} = \left|\dfrac{1}{x^{2}_{n}} - z\right|$.
    Posso calcolare $\sqrt{z}$ a una data precisione: ``prosegui a calcolare nuovi termini della successione $x_{n}$
    \textbf{finché l'errore non è più piccolo di $10^{-3}$}''.


  \end{itemize}

\end{frame}

\begin{frame}[fragile]\frametitle{Ciclo While (I)}
  \textbf{Finché} la condizione è \textbf{vera} esegui una certa operazione (ciclo \textbf{\texttt{while-do}}
  o semplicemente \textbf{\texttt{while}}):
  \begin{columns}[T]
    \begin{column}[T]{5cm}
      Pseudocodice:
      \begin{algorithmic}[1]     
	\State $s \gets 0$
	\State $n \gets 1$
	\While{$n \leq 100$}
	  \State $s \gets s + n$
	  \State $n \gets n + 1$
	\EndWhile
      \end{algorithmic}
    \end{column}
    \begin{column}[T]{5cm}
      Java:
      \begin{JavaCodePlain}[commandchars=\\!|]
	int s = 0;
	int n = 1;
	\Word!while| (\Green!n <= 100|) {
	  \Red!s = s + n;|
	  \Red!n = n + 1;|
	}
      \end{JavaCodePlain}
    \end{column}
  \end{columns}
  ${}$\\
  Nel momento in cui la condizione all'interno del while diventa falsa si \textbf{esce} dal ciclo.
\end{frame}

\begin{frame}[fragile]\frametitle{Ciclo While (II)}
  \textbf{Finché} la condizione è \textbf{vera} esegui una certa operazione (ciclo \textbf{\texttt{while-do}}):
  \begin{columns}[T]
    \begin{column}[T]{5cm}
      Pseudocodice:
      \begin{algorithmic}[1]     
	\While{condizione}
	  \State comandi
	\EndWhile
      \end{algorithmic}
    \end{column}
    \begin{column}[T]{5cm}
      Java:
      \begin{JavaCodePlain}[commandchars=\\!|]
	\Word!while| (\Green!condizione|) {
	  \Red!comandi;|
	}
      \end{JavaCodePlain}
    \end{column}
  \end{columns}
  ${}$\\
  Nel momento in cui la condizione all'interno del while diventa falsa si \textbf{esce} dal ciclo.
\end{frame}

\begin{frame}{Ciclo While (III)}

  % Define block styles
  \tikzstyle{decision} = [diamond, draw, fill=green!20,text width=4.5em, text
                          badly centered, node distance=3cm, inner sep=0pt]
  \tikzstyle{block} = [rectangle, draw, rounded corners]
  \tikzstyle{line} = [draw, -latex']

  Diagrama di flusso dell'istruzione While:
  \begin{center}
    \begin{tikzpicture}[node distance = 1cm, auto]
	% Place nodes
	\node [block, text width=1.75cm, fill=blue!20!white] (init) {\texttt{int n = 1;}\\\texttt{int s = 0;}};
	\node [draw,circle,minimum size=1mm,inner sep=0pt,outer sep=0pt,fill=black, below of=init] (vuoto) {};
	\node [decision, below of=init] (decide) {$n \leq 100$};
	\node [block, below of=decide, node distance=2cm] (stop) {resto del programma};
	\node [block, right of=decide, node distance=4cm, fill=red!50, text width=1.75cm] (comandi) {\texttt{s = s + n;}\\ \texttt{n = n + 1;}};
	% Draw edges
	\path [line] (init) -- (decide);
	\path [draw] (comandi)  --++ (0,2) |- (vuoto);
	\path [line] (decide) -- node {sì}(comandi);
	\path [line] (decide) -- node {no}(stop);
    \end{tikzpicture}   
  \end{center}

\end{frame}

\begin{frame}[fragile]\frametitle{Ciclo While (IV)}

  Esecuzione del ciclo passo-passo:
  \begin{table}[]
  \centering
    \begin{tabular}{|c|c|c|c|}
    \hline
    \textbf{step} & \textbf{$n$} & \textbf{$s$} & \textbf{$n \leq 100 $?} \\ \hline
      $-$  & $1$   & $0$    & $-$              \\ \hline
      1    & $1$   & $0$    & sì               \\ \hline
      2    & $2$   & $1$    & sì               \\ \hline
      3    & $3$   & $3$    & sì               \\ \hline
      4    & $4$   & $6$    & sì               \\ \hline
      ...  & ...   & ...    & ...              \\ \hline
      99   & $99$  & $4851$ & sì               \\ \hline
      100  & $100$ & $4950$ & sì               \\ \hline
      101  & $101$ & $5050$ & no               \\ \hline
% %       $-$  & $101$ & $5050$ &                  \\  \hline   
    \end{tabular}
  \end{table}

\end{frame}

\begin{frame}[fragile]\frametitle{Ciclo While (VI)}
  Attenzione agli estremi dei cicli:
  \begin{columns}[T]
    \begin{column}[T]{5cm}
      \begin{JavaCodePlain}[commandchars=\\!|]
	int n = 1;
	\Word!while| (\Green!n <= 10|) {
	  \Word!println|(\String!"Ciao, mondo\bang"|); 
	  n = n + 1; 
	}
      \end{JavaCodePlain}
      Scrive \texttt{\String{Ciao, mondo!}} \textbf{10} volte.
    \end{column}
    \begin{column}[T]{5cm}
      \begin{JavaCodePlain}[commandchars=\\!|]
	int n \Red!= 0|;
	\Word!while| (\Green!n \Red!< 10||) {
	  \Word!println|(\String!"Ciao, mondo\bang"|); 
	  n = n + 1; 
	}
      \end{JavaCodePlain}
      Scrive \texttt{\String{Ciao, mondo!}} \textbf{10} volte.
    \end{column}
  \end{columns}
\end{frame}

\begin{frame}[fragile]\frametitle{Ciclo While (VI)}
  Se la condizione non diventa mai falsa, allora il ciclo \textbf{non termina mai}.
  \begin{columns}[T]
    \begin{column}[T]{5cm}
      \begin{JavaCodePlain}[commandchars=\\!|]
	int s = 0;
	int n = 99;
	\Word!while| (\Green!n \bang= 0|) {
	  s = s + n; 
	  \Green!// n = 99, \dots, 1,\Red!-1, \dots||
	  n = n - 2; 
	}
      \end{JavaCodePlain}
    \end{column}
    \begin{column}[T]{5cm}
      \begin{JavaCodePlain}[commandchars=\\!|]
	\Red!int p = 0;|
	int n = 1;
	\Word!while| (\Green!p < 100|) {
	  p = p * n; \Green!// p == 0|
	  n = n + 1;
	}
      \end{JavaCodePlain}
    \end{column}
  \end{columns}
\end{frame}

\begin{frame}[fragile]\frametitle{Ciclo While (VII)}
  \begin{columns}[T]
    \begin{column}[T]{5cm}
      Se la condizione è \textbf{sempre vera}, allora il ciclo non termina mai:
      \begin{JavaCodePlain}[commandchars=\\!|]
	\Word!while| (\Green!\textbf!true||) {
	  s = s + n; 
	  n = n + 1; 
	}
      \end{JavaCodePlain}
      Il comando $\Word{\textbf{break}}$ permette di uscire dal ciclo.
      \begin{JavaCodePlain}[commandchars=\\!|]
	int s = 0;
	int n = 1;
	\Word!while| (\Green!n <= 50|) {
	  \Green!// se n = 33 esco|
	  \Word!if| (n == 33) {
	    \Word!break|;
	  }
	  s = s + n;
	  n = n + 1;
	}
      \end{JavaCodePlain}

    \end{column}
    \begin{column}[T]{5cm}
      Se devo saltare dei valori allora posso usare il comando $\Word{\textbf{continue}}$:
      \[ s = \sum_{i=1, i \neq 3}^{50} i \]
      \begin{JavaCodePlain}[commandchars=\\!|]
	int s = 0;
	int n = 1;
	\Word!while| (\Green!n <= 50|) {
	  \Green!// salto il caso n = 3|
	  \Word!if| (n == 3) {
	    n = n + 1;
	    \Word!continue|;
	  }
	  s = s + n;
	  n = n + 1;
	}
      \end{JavaCodePlain}
    \end{column}
  \end{columns}
\end{frame}

\begin{frame}{Nuovo progetto Eclipse}
  \begin{enumerate}
   \item Aprire Eclipse
   \item \emph{File} $\rightarrow$ \emph{New} $\rightarrow$ \emph{Project} $\rightarrow$ \emph{Java Project}
   \item Inserire il nome e click su \emph{Finish}
   \item Click destro su \emph{src} $\rightarrow$ \emph{New} $\rightarrow$ \emph{Class\dots}
   \item Inserire il nome e check su \newline
	  \texttt{public static void main(String[] args)}
   \item Click su \emph{Finish}
  \end{enumerate}
  Se avete difficoltà o problemi scrivetemi una mail!

\end{frame}

\begin{frame}{Esercizi (I)}
  \begin{enumerate}
   \item Scrivete un programma che stampi la stringa \texttt{\String{Ciao, mondo!}} a schermo per 10 volte;
   \item Scrivere un programma che stampi tutti i numeri pari fino a 1000;
  \end{enumerate}
\end{frame}

\begin{frame}[fragile]\frametitle{Ciclo Do-While (I)}
  Esegui una certa operazione \textbf{finché} la condizione è \textbf{vera} (ciclo \textbf{\texttt{do-while}}):
  \begin{columns}[T]
    \begin{column}[T]{5cm}
      Pseudocodice:
      \begin{algorithmic}[1]     
	\State $num \gets 20$
	\State $count \gets 0$
	\Do
	  \State $num \gets num / 2$
	  \State $count \gets count + 1$
	\DoWhile{$num \; \mathit{mod} \; 2 \neq 0$}
      \end{algorithmic}
    \end{column}
    \begin{column}[T]{5cm}
      Java:
      \begin{JavaCodePlain}[commandchars=\\!|]
	int num = 20;
	int count = 0;
	\Word!do| {
	  num = num / 2;
	  count = count + 1;
	} \Word!while| (\Green!num \% 2 \bang= 0|)
      \end{JavaCodePlain}
    \end{column}
  \end{columns}
  ${}$\newline
  La differenza fondamentale tra la forma \texttt{while} e quella \texttt{do-while} 
  è che con il \texttt{do-while} i comandi all'interno del ciclo vengono 
  \textbf{sempre} eseguiti \textbf{almeno una volta}.
\end{frame}

\begin{frame}{Ciclo Do-While (II)}

  % Define block styles
  \tikzstyle{decision} = [diamond, draw, fill=green!20,text width=5.5em, text
                          badly centered, node distance=2cm, inner sep=0pt, font=\scriptsize]
  \tikzstyle{block} = [rectangle, draw, rounded corners]
  \tikzstyle{line} = [draw, -latex']

  Diagrama di flusso dell'istruzione Do-While:
  \begin{center}
    \begin{tikzpicture}[node distance = 1cm, auto]
	% Place nodes
	\node [block, text width=1.75cm, fill=blue!20!white, text width=3.5cm] (init) {\texttt{int num = 20;}\\\texttt{int count = 0;}};
	\node [block, below of=init, fill=red!50, text width=3.5cm, node distance=1.5cm] (comandi) {\texttt{num = num / 2;}\\ \texttt{count = count + 1;}};
	% \node [draw,circle,minimum size=1mm,inner sep=0pt,outer sep=0pt,fill=black, below of=init] (vuoto) {};
	\node [decision, below of=comandi] (decide) {\texttt{\textbf{while}}\\$num \; \mathit{mod} \; 2 \neq 0$};
	\node [block, below of=decide, node distance=2cm] (stop) {resto del programma};
	% Draw edges
	\path [line] (init) -- (comandi);
	\path [line] (comandi) -- (decide);
	\path [line] (decide) -- node {no}(stop);
	\path [line] (decide)  --++ (3,0) node[anchor=south,pos=0.5]{sì} |- (comandi);
    \end{tikzpicture}   
  \end{center}

\end{frame}


\section{Ciclo For}
\begin{frame}[fragile]\frametitle{Ciclo For (I)}
  Il ciclo \textbf{\texttt{for}} è un altro tipo di interazione dove si scorre una variabile,
  detta \textbf{indice} entro un intervallo di valori, con un dato \textbf{incremento}

  \begin{JavaCodePlain}[commandchars=\\!|]
    \Word!for| (\Blue!inzializzazione;| \Green!condizione| \Violet!incremento|) {
      comandi;
    }
  \end{JavaCodePlain}

  Esempio:
  \begin{JavaCodePlain}[commandchars=\\!|]
    int s = 0;
    \Word!for| (\Blue!int n = 1;| \Green!n <= 100;| \Violet!n++|) {
      s = s + n;
    }
  \end{JavaCodePlain}

\end{frame}

\begin{frame}[fragile]\frametitle{Ciclo For (II)}
  Il ciclo \textbf{\texttt{for}} è un altro tipo di interazione dove si scorre una variabile,
  detta \textbf{indice} entro un intervallo di valori, con un dato \textbf{incremento}

  \begin{JavaCodePlain}[commandchars=\\!|]
    \Word!for| (\Blue!inzializzazione;| \Green!condizione;| \Violet!incremento|) {
      comandi;
    }
  \end{JavaCodePlain}

  Note:
  \begin{itemize}
   \item nell'\texttt{\Blue{inzializzazione;}} potete \textbf{dichiarare} una variabile;
   \item Per l'\texttt{\Violet{incremento;}} potete usare step anche diversi da 1;
  \end{itemize}

\end{frame}

\begin{frame}[fragile]\frametitle{Ciclo For (III)}
  Quando la condizione è \textbf{falsa} il ciclo viene interrotto:
  \begin{columns}[T]
    \begin{column}[T]{5cm}
      Pseudocodice:
      \begin{algorithmic}[1]     
	\State $s \gets 0$
	\For{$n \gets 1$ to $100$ by $1$}
	  \State $s \gets s + n$
	\EndFor
      \end{algorithmic}
    \end{column}
    \begin{column}[T]{6cm}
      Java:
      \begin{JavaCodePlain}[commandchars=\\!|]
	int s = 0;
	\Word!for| (\Blue!int n = 1;| \Green!n <= 100;| \Violet!n++|) {
	  s = s + n;
	}
      \end{JavaCodePlain}
    \end{column}
  \end{columns}
\end{frame}

\begin{frame}{Ciclo For (IV)}

  % Define block styles
  \tikzstyle{decision} = [diamond, draw, fill=green!20,text width=3.5em, text
                          badly centered, node distance=1.5cm, inner sep=0pt, font=\scriptsize]
  \tikzstyle{block} = [rectangle, draw, rounded corners]
  \tikzstyle{line} = [draw, -latex']

  Diagrama di flusso dell'istruzione for:
  \begin{center}
    \begin{tikzpicture}[node distance = 1cm, auto]
	% Place nodes
	\node [block, fill=blue!20!white]                                   (init)      {\texttt{int s = 0;}};
	\node [block, below of=init, fill=blue!5!white, node distance=1cm] (forinit)   {\texttt{int n = 1;}};
	\node [decision, below of=forinit]                                  (decide)    {$n \leq 100$};
	\node [block, fill=red!50, below of=decide, node distance=1.5cm]    (comandi)   {\texttt{comandi;}};
	\node [block, fill=violet!30!white, below of=comandi]               (incremento){\texttt{i++}};
	\node [circle, below of=incremento]         (vuoto) {};
	\node [block, right of=vuoto, node distance=3cm]                    (stop) {resto del programma};
	% Draw edges
	\path [line] (init)       -- (forinit);
	\path [line] (forinit)    -- (decide);
	\path [line] (decide)     --  node {sì}(comandi);
	\path [line] (comandi)    -- (incremento);
	\path [line] (decide)  --++ (3,0) node[anchor=south,pos=0.5]{no} -| (stop);
	\path [line] (incremento)  --++ (0,-1) --++ (-2,0)  |- (decide);
    \end{tikzpicture}   
  \end{center}

\end{frame}

\begin{frame}[fragile]\frametitle{Esempi di ciclo \texttt{for}}

  \begin{itemize} 
    \item Nel ciclo $i$ assume i valori $i = 0, 5, 10, 15, \dots$
  \end{itemize}
  \begin{JavaCodePlain}[commandchars=\\!|]
    \Word!for| (\Blue!int i = 0;| \Green!i < 100;| \Violet!i = i + 5|) {
      comandi;
    }
  \end{JavaCodePlain}

  \begin{itemize} 
    \item Nel ciclo $j$ assume i valori $j = 1, 2, 4, \dots$
  \end{itemize}  
  \begin{JavaCodePlain}[commandchars=\\!|]
    \Word!for| (\Blue!int j = 1;| \Green!j < 1024;| \Violet!j = j * 2|) {
      comandi;
    }
  \end{JavaCodePlain}
  
  \begin{itemize} 
    \item Nel ciclo $k$ assume i valori $k = 10, 9, 8, \dots$
  \end{itemize}    
  \begin{JavaCodePlain}[commandchars=\\!|]
    \Word!for| (\Blue!int k = 10;| \Green!k > 0;| \Violet!k--|) {
      comandi;
    }
  \end{JavaCodePlain}

  Esistono gli operatori \texttt{+=}, \texttt{-=}, \texttt{*=}, ma vi consiglio di scrivere
  l'incremento in modo espicito per evitare errori.

\end{frame}

\begin{frame}[fragile]\frametitle{Analogie tra ciclo \texttt{for} e ciclo \texttt{while}}

  I cicli \textbf{\texttt{for}} e \textbf{\texttt{while}} sono equivalenti. Tutti i cicli \textbf{\texttt{for}} 
  possono essere ``tradotti'' in cicli \textbf{\texttt{while}} e \textbf{viceversa}.
  
  \begin{JavaCodePlain}[commandchars=\\!|]
    \Word!for| (\Blue!inzializzazione;| \Green!condizione;| \Violet!incremento|) {
      comandi;
    }
  \end{JavaCodePlain}

  \begin{JavaCodePlain}[commandchars=\\!|]
    \Blue!inzializzazione;|
    \Word!for| (\Green!condizione|) {
      comandi;
      \Violet!incremento;|
    }
  \end{JavaCodePlain}
\end{frame}


\section{Esericizi}
\subsection[Esercizi]{Esercizi}

\begin{frame}{Esercizi (I)}
  \begin{itemize}
   \item Scrivete un programma che stampi la canzone popolare inglese ``\emph{99 bottiglie di birra}''
   \item (vedete anche \url{https://esolangs.org/wiki/99_bottles_of_beer})
  \end{itemize}
  \begin{quote}
   «99 bottles of beer on the wall, 99 bottles of beer.\newline
   Take one down, pass it around, 98 bottles of beer on the wall \newline
   99 bottles of beer on the wall, 99 bottles of beer.\newline
   Take one down, pass it around, 98 bottles of beer on the wall  \newline
   ...\newline
   1 bottle of beer on the wall, 1 bottle of beer.\newline
   Take one down, pass it around, no more bottles of beer on the wall\newline
   There no more bottles of beer on the wall, no more bottles of beer.»
  \end{quote}

\end{frame}

\begin{frame}{Esercizi (II)}
  \begin{itemize}
    \item Utilizzando il ciclo \texttt{while} scrivete un programma che dato un 
    intero stampi a schermo la ``tabellina''.
    Ad esempio, se il numero è 7 dovrete stampare a schermo:
    \begin{itemize}
      \item 7*0 = 0
      \item 7*1 = 7
      \item 7*2 = 14
      \item $\dots$
      \item 7*10 = 70
    \end{itemize}
    \item riscrivete il programma precendente usando il ciclo \texttt{for}.
  \end{itemize}
\end{frame}


\begin{frame}{Esercizi (III)}
  \begin{itemize}
   \item Scrivete un programma che calcoli il fattoriale di un numero intero a vostra scelta.
  \end{itemize}

  La definizione del fattoriale è la seguente:
  \begin{equation}
    n! = n \times (n - 1) \times \dots \times 1
  \end{equation}
  quindi il calcolo del fattoriale può essere definito da: 
  \begin{center}
    \begin{minipage}{8cm}
      \begin{algorithmic}[1]
	\State $fatt \gets ?$ \Comment Quale valore va messo qui?
	\For{$i \gets 1$ to $N$}
	  \State $fatt \gets fatt \times i$ 
	\EndFor
      \end{algorithmic}
   \end{minipage}
  \end{center}

\end{frame}

\begin{frame}{Esercizi (IV)}
  \begin{itemize}
    \item Scrivere un programma che stampi i valori della serie di Fibonacci minori di 10000.
    La serie di Fibonacci \`e definita da:
    \begin{equation*}
      \left\{\begin{aligned}
	  & x_0 = 1\\
	  & x_1 = 1\\
	  & x_{n+1} = x_{n} + x_{n-1}
      \end{aligned}\right.
    \end{equation*}
  \end{itemize}

\end{frame}


\begin{frame}{Esercizi (V)}
  \begin{itemize}
    \item Scrivere un programma che usi il metodo per il calcolo della radice quadrata di Newton.
    
    \begin{equation*}
      \left\{\begin{aligned}
	  & x_0 = 0.5\\
	  & x_{n +1} = 0.5 \cdot {x_n} (3  - zx^2_n)
      \end{aligned}\right.
    \end{equation*}
    \begin{equation*}
     \lim_{n \to \infty} x_{n} = \sqrt{z}
    \end{equation*}
  \end{itemize}
  
  Il programma deve calcolare la serie definita sopra fino a che l'errore $\varepsilon_{n} = |x^{2}_{n} - z|$,
  non è più piccolo di $10^{-3}$. Per il valore assoluto utilizzate la funzione \texttt{Math.\Blue{abs()}}.

\end{frame}

\begin{frame}{Esercizi (VI)}
  \begin{itemize}
    \item Metodo della bisezione
  \end{itemize}
  \begin{tiny}
    \url{https://ece.uwaterloo.ca/~dwharder/NumericalAnalysis/10RootFinding/bisection/bisection.gif}
  \end{tiny}

  \begin{scriptsize}
    \begin{center}
      \begin{minipage}{8cm}
	\begin{algorithmic}[1]
	  \Require Function $f$, endpoint values $a, b$, tolerance $\varepsilon$, maximum iterations $N_{MAX}$
		   $a < b$, either $f(a) < 0$ and $f(b) > 0$ or $f(a) > 0$ and $f(b) < 0$
	  \Ensure value which differs from $a$ root of $f(x)=0$ by less than $\varepsilon$
    
	  \State $N \gets 1$
	  \While{$N \leq N_{MAX}$}
	    \State $c \gets (a + b)/2$
	    \If{$f(c) = 0 \lor (b – a)/2 < \varepsilon$}
	      \State \texttt{\textbf{print}}(c)
	      \State \Return;
	    \EndIf
	    \State $N \gets N + 1$
	    \If{$sign(f(c)) = sign(f(a))$}
	      \State $a \gets c$
	    \Else 
	      \State $b \gets c$
	    \EndIf
	  \EndWhile
	  \State \texttt{\textbf{print}}(Non ho trovato risultati)
	\end{algorithmic}
      \end{minipage}
    \end{center}
  \end{scriptsize}

\end{frame}

\begin{frame}{Esercizi (VII)}
  Test di primalità
  \begin{itemize}
    \item Scrivere un programma che, dato un intero positivo, verifichi se quel numero è primo oppure no.
  \end{itemize}

   Un numero $n \in \mathbb{N}, n > 1$ \`e \textbf{primo} se e solo se \`e divisibile solo per $1$ e per
   se stesso.
\end{frame}
% \appendix
% \section*{Backup}
% \begin{frame}
\begin{Huge}
Backup 
\end{Huge}
\end{frame}


\end{document}
