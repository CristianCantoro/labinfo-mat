\begin{frame}[fragile]\frametitle{Esercizi (I)}

  \begin{itemize}
   \item Scrivere un  programma che apra il file  “esercizio01.txt”  (scaricatelo da  qui
   {\footnotesize \LightBlue{\url{http://bit.ly/esercizi-labinfo}}} e  inseritelo nella
   cartella principale del  progetto)
   \item Il  file  contiene una serie di  array  di  interi,  la  prima  riga indica il numero
   di elementi per riga, mentre le  righe successive sono effettivamente gli array.  Stampate
   a video il massimo di ciascuna riga e  alla fine il valore massimo totale.
  \end{itemize}

  \begin{center}
  \begin{minipage}[c]{4cm}
  Esempio di input:
  \begin{JavaCodePlain}[commandchars=\\!|]
  8
  1  2  9  5  7  12  2  3
  4  9  1  5  8  14  5  6
  0  6  11 5  9  2   6  0   
  \end{JavaCodePlain}
  \end{minipage}
  \end{center}
\end{frame}

\begin{frame}[fragile]\frametitle{Esercizi (II)}
  \begin{itemize}
   \item Scrivere un  programma che apra il file  "divina.txt" (scaricatelo da
   {\footnotesize \LightBlue{\url{http://bit.ly/esercizi-labinfo}}} e inseritelo
   nella cartella principale del  progetto);
   \item Il file contiene il primo  capitolo della divina commedia.  Leggete linea per linea, contando
   il numero di \String{"a"},  \String{"e"},  \String{"i"},  \String{"o"},  \String{"u"};
   \item Una volta letto tutto il file,  create  un  nuovo file  chiamato “risultati.txt” in  cui  scriverete il
   numero di \String{"a"},  \String{"e"},  \String{"i"},  \String{"o"},  \String{"u"} presenti;
  \end{itemize}

  Suggerimento:
  \begin{itemize}
   \item utilizzate 5  variabili diverse  o  un  array  da  5  elementi,  come  preferite)
  scorrete la  stringa e controllate carattere per  carattere con il metodo
  \end{itemize}

  \begin{center}
  \begin{minipage}[c]{5cm}
  \begin{JavaCodePlain}[commandchars=\\!|]
  nomeStringa.charAt(indice)
  \end{JavaCodePlain}
  \end{minipage}
  \end{center}
  
\end{frame}

