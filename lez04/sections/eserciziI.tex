\begin{frame}{Esercizi (parte I) (I)}

  Scrivere un programma in cui:
  \begin{itemize}
   \item si inserire un numero intero $N \leq 10$ da console;
   \item Inizializzare un  array  di  interi di lunghezza $N$ con  
   i valori presi in input  da  console (ciclare finché non si ottengono $N$ valori
   del tipo desiderato);
   \item Dopo aver letto tutti i valori stampare progressivamente i valori inseriti;
  \end{itemize}
\end{frame}
  
\begin{frame}[fragile]\frametitle{Esercizi (parte I) (II)}  
  Per esempio, se  l'array inserito è  il  seguente  $\{2,  5,  7,  4\}$. Il  programma deve ciclare
  sull’array stampando:
  \begin{center}
  \begin{JavaCodePlain}[commandchars=\\!|]
      2
      2, 5
      2, 5, 7
      2, 5, 7, 4
  \end{JavaCodePlain}   
  \end{center}

  \begin{itemize}
   \item Suggerimento: sono due  \Jfor{} annidati (uno all'interno dell’altro) utilizzate
   \texttt{System.out.}\JPrint{} per  stampare i valori e quando uscite dal \Jfor{}
   interno stampate una riga vuota
  \end{itemize}


\end{frame}
