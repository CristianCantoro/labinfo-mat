
\begin{frame}{Scrittura file (I)}
  In modo analogo a quanto avviene per la lettura da file, per la scrittura:
  \begin{enumerate}[<+->]
   \item dobbiamo creare un oggetto che rappresenti il file su cui vogliamo scrivere,
   esso rappresenta il file nel \emph{filesystem};
   \item utilizziamo la classe di Java \alert{\texttt{FileWriter}} , questa classe si
   preoccupa di preparare il file per la scrittura gestendo, per esempio, l'\textbf{encoding} del file.
   \item creiamo un ``buffer'' \alert{\texttt{BufferedWriter}} che effettivamente conterrà il 
   testo da scrivere sul file.
   \item per scrivere usiamo il metodo \alert{\texttt{write()}} del buffer
   \item Quando abbiamo terminato, chiudiamo il buffer.
  \end{enumerate}
  
\end{frame}

\begin{frame}[fragile]\frametitle{Scrittura file (II)}

  \begin{JavaCodePlain}[commandchars=\\!|]
  \Jimport java.io.BufferedWriter;
  \Jimport java.io.File;
  \Jimport java.io.FileWriter;
  \Jimport java.io.IOException;

  \Jpublic \Jclass ScritturaFile {
    \Jpublic \Jstatic \Jvoid main(String[] args) throws IOException {

      String text = \String!"TestoEsempio"|;

      File file = \Jnew File(\String!"esempio.txt"|);
      FileWriter fw = \Jnew FileWriter(file);
      BufferedWriter bw = \Jnew BufferedWriter(fw);

      bw.write(text);
      bw.flush();
      bw.close();
      \dots
  \end{JavaCodePlain}
\end{frame}

\begin{frame}[fragile]\frametitle{Scrittura file (III)}

  Ogni qual volta aggiungiamo una delle classi necessarie alla scrittura
  su File Java segnala un errore che può essere risolto aggiungengo
  gli \Word{import} \emph{statement} necessari.

  \begin{JavaCodePlain}[commandchars=\\!|]
  \Jimport java.io.BufferedWriter;
  \Jimport java.io.File;
  \Jimport java.io.FileWriter;
  \Jimport java.io.IOException;
  \end{JavaCodePlain}
\end{frame}

\begin{frame}[fragile]\frametitle{Scrittura file (V)}

  Anche quando aggiungiamo il \texttt{FileWriter} dobbiamo preparci a gestire
  l'eccezione, in questo caso segnaliamo il fatto che il main può generare
  una \texttt{IOException}.

  \begin{JavaCodePlain}[commandchars=\\!|]

  \Jpublic \Jclass ScritturaFile {

    \Jpublic \Jstatic \Jvoid main(String[] args) \textbf!\Jthrows IOException| {

      String text = \String!"TestoEsempio"|;

      File file = \Jnew File(\String!"esempio.txt"|);
      FileWriter fw = \Jnew FileWriter(file);
      BufferedWriter bw = \Jnew BufferedWriter(fw);

  \end{JavaCodePlain}
\end{frame}

\begin{frame}[fragile]\frametitle{Scrittura file (VI)}

  \begin{JavaCodePlain}[commandchars=\\!|]

  \Jpublic \Jclass ScritturaFile {

    \Jpublic \Jstatic \Jvoid main(String[] args) \Jthrows IOException {

      String text = \String!"TestoEsempio"|;

      File file = \Jnew File(\String!"esempio.txt"|);
      FileWriter fw = \Jnew FileWriter(file);
      BufferedWriter bw = \Jnew BufferedWriter(fw);

      bw.write(text);
      \Jcomment!Forza lo svuotamento del buffer e la scrittura su|
      \Jcomment!file|
      bw.flush();
      bw.close();
    }
  }
  \end{JavaCodePlain}
\end{frame}
