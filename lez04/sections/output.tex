
\begin{frame}{Scrittura su \texttt{stdout} (I)}
  Abbiamo già visto dei comandi/funzioni per scrivere sullo standard output (``a schermo''):
  \begin{itemize}
   \item \texttt{System.out.\LightBlue{print}( \String{stringa} )}
   \item \texttt{System.out.\LightBlue{prinlnf}( \String{stringa} )}
   \item \texttt{System.out.\LightBlue{printf}( \String{formato}, argomenti, \dots)}
  \end{itemize}
 
\end{frame}

\begin{frame}{Scrittura su \texttt{stdout} (II)}

Alcune cose utili:
  \begin{itemize}
   \item Con \LightBlue{printf} il formato \String{"\%nd"}, dove $n$ è un numero (intero),
   stampa spazi aggiuntivi in modo tale che il numero occupi sempre $n$ spazi (si veda 
   l'esercizio \texttt{MatriceTrasposta});
   \item Con \LightBlue{printf} il formato \String{"\%0nd"}, dove $n$ è un numero (intero),
   stampa degli zeri aggiuntivi in modo tale che il numero occupi sempre $n$ spazi, ad
   esempio \String{"\%03d"} stampa il numero \texttt{7} come \texttt{007};
   \item altre \emph{sequenze di escape} particolari sono \texttt{\string\t} per inserire una
   tabulazione (un numero di spazi variabile che allinea l'output lungo una data colonna)
  \end{itemize}
\end{frame}
