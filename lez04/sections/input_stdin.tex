
\begin{frame}[fragile]\frametitle{Lettura input da tastiera (I)}

  Abbiamo già visto che per stampare dei valori a schermo si usa:
  \begin{itemize}
    \item \texttt{\textbf{System.out}}
  \end{itemize}
  questo permette di interagire con lo \texttt{stdout}.
  
  ${}$

  Analogamente:
  \begin{itemize}
    \item \texttt{\textbf{System.in}}
  \end{itemize}
  permette di interagire con lo \texttt{stdin}.
  
%   ${}$
% 
%   Nella libreria \texttt{java.util} di Java esiste l'oggetto \texttt{\textbf{Scanner}} 
%   che ci permette di utilizzare l’input della console:
%   \begin{JavaCodePlain}[commandchars=\\!|]
%   Scanner input = new Scanner(System.in);
%   \end{JavaCodePlain}     

\end{frame}

\begin{frame}[fragile]\frametitle{Lettura input da tastiera (II)}
  Nella libreria \texttt{java.util} di Java esiste l'oggetto \texttt{\textbf{Scanner}} 
  che ci permette di utilizzare l’input della console:
  
  \begin{JavaCodePlain}[commandchars=\\!|]
  \Jimport java.util.Scanner;

  \Jpublic \Jclass LetturaFile {

    \Jpublic \Jstatic \Jvoid main(String[] \Jargs) {
      Scanner input = \Jnew Scanner(\JsystemIn);
    
      \dots
      
      \Jcomment!Al termine della lettura chiudere il canale di input|
      input.close();
    }
  }
  \end{JavaCodePlain}     

\end{frame}

\begin{frame}[fragile]\frametitle{Lettura input da tastiera (III)}
  
  Per potere utilizzare \texttt{\textbf{Scanner}} dobbiamo \emph{importare}
  la libreria \texttt{\textbf{java.util}} aggiungendo la riga:
  \begin{JavaCodePlain}[commandchars=\\!|]
  \Jimport java.util.Scanner;
  \end{JavaCodePlain}
  all'inizio del file.
  
  ${}$

  Lo statement \Word{import} è il modo in cui si possono \emph{importare} librerie in
  un programma Java.

  ${}$

  Una libreria è un insieme di funzioni e/o strutture dati predefinite e predisposte
  per essere riutilizzate facilmente all'interno di svariati programmi.

\end{frame}

\begin{frame}[fragile]\frametitle{Modalità di lettura da \texttt{stdin} (I)}

  Dopo aver dichiarato l'oggetto di tipo \texttt{Scanner}
  \begin{JavaCodePlain}[commandchars=\\!|]
  Scanner input = \Jnew Scanner(\JsystemIn);
  \end{JavaCodePlain}

  si possono usare i seguenti \textbf{metodi}:
  \begin{itemize}
   \item \texttt{input.nextInt()} \structure{$\Rightarrow$} lettura di un \Jint;
   \item \texttt{input.nextDouble()} \structure{$\Rightarrow$} lettura di un \Jdouble;
   \item \texttt{input.next()} \structure{$\Rightarrow$} lettura di una stringa;
   \item \texttt{input.nextLine()} \structure{$\Rightarrow$} legge l'intera linea fino al
	 carattere \texttt{newline} (Invio);
  \end{itemize}

\end{frame}

\begin{frame}[fragile]\frametitle{Modalità di lettura da \texttt{stdin} (II)}

  Se viene inserito un valore di un tipo non corrispondendente viene causato
  un \textbf{errore} ovvero viene \emph{lanciata un'\textbf{eccezione}}.

  ${}$\\
  
  \pause{
  Se per esempio uso \textbf{\texttt{input.nextInt()}}:
  \begin{JavaCodePlain}[commandchars=\\!|]
  Inserisci un numero intero: \Green!3.2|
  \end{JavaCodePlain}

  \begin{redtext}
  \begin{JavaCodePlain}[commandchars=\\!|]
  \textbf!Exception| in thread "main" java.util.InputMismatchException
    at java.util.Scanner.throwFor(Scanner.java:864)
    at java.util.Scanner.next(Scanner.java:1485)
    at java.util.Scanner.nextInt(Scanner.java:2117)
    at java.util.Scanner.nextInt(Scanner.java:2076)
    at LetturaIntero.main(LetturaIntero.java:11)
  \end{JavaCodePlain}
  \end{redtext}
  }

\end{frame}

\begin{frame}[fragile]\frametitle{Modalità di lettura da \texttt{stdin} (III)}

  Esistono dei metodi per controllare il tipo di dato inserito:
  \begin{itemize}
   \item \texttt{input.hasNextInt()} \structure{$\Rightarrow$} controlla che venga letto un \Jint;
   \item \texttt{input.hasNextDouble()} \structure{$\Rightarrow$} controlla che venga letto un \Jdouble;
   \item etc.
  \end{itemize}

  Queste funzioni ritornano \Word{true} se il prossimo valore è del tipo indicato:
  \begin{JavaCodePlain}[commandchars=\\!|]
    \Jprint!"Inserisci la tua altezza (in cm): "|;
    \Jwhile (\bang!|input.hasNextInt()) {
      \Jprintln!"Il numero inserito non è un intero valido."|;
      \Jprint!"Inserisci la tua altezza (in cm): "|;
      input.next();
    }   
  \end{JavaCodePlain}


\end{frame}

