\subsection[Definizione]{Definizione}

\begin{frame}{Istruzione \texttt{if}}
     Le \textbf{istruzioni condizionali} permettono di effettuare operazioni diverse a seconda 
     dei valori delle variabili.
    \begin{algorithmic}[1]
      \If{\alert{condizione}}
	\State istruzione 1
      \Else
	\State istruzione 2
      \EndIf
    \end{algorithmic}
    \alert{condizione} deve essere una \textbf{espressione booleana}.
\end{frame}


\begin{frame}{Istruzione \texttt{if}}
    
    In Java: \\
    \begin{center}
      \texttt{     
	\textbf{if} (\alert{condizione}) \{ \newline
	      comando1 \newline
	    \} \textbf{else} \{ \newline
	      comando2 \newline
	    \} \newline}
    \end{center}

\end{frame}