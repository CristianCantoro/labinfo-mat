\subsection[Informazioni generali]{Informazioni generali}

\begin{frame}{Chi sono}
  Cristian Consonni

  \begin{itemize}
    \item \textbf{DISI - Dipartimento di Ingegneria e Scienza dell'Informazione}
    \item \textbf{Pagina web} del laboratorio: \structure{\url{http://disi.unitn.it/~consonni/teaching}}
    \item \textbf{Email}: \structure{\url{cristian.consonni@unitn.it}}
    \item \textbf{Ufficio}: Povo 2 - Open Space 9
      \begin{itemize}
	\item Per domande: scrivetemi una mail
	\item Ricevimento: su appuntamento via mail
      \end{itemize}
  \end{itemize}
\end{frame}



\begin{frame}{Obiettivi del laboratorio}
  Obiettivi del laboratorio:
  \begin{itemize}
    \item Apprendere i fondamenti di un vero linguaggio di programmazione (Java)
    \item Svolgere il progetto
  \end{itemize}

  Obiettivi del laboratorio
  \begin{enumerate}
    \item Fare esperienza in laboratorio
    \item Raggiungere una buona manualità nell'uso degli strumenti standard
    \item Esercizi
  \end{enumerate}

\end{frame}

\pgfdeclareimage[width=0.6\paperwidth]{xkcd}{img/11th_grade.png}
\begin{frame}{Manualità (I)}
  \begin{center}
    \pgfuseimage{xkcd}
  \end{center}
  \url{https://xkcd.com/519/}
\end{frame}

\pgfdeclareimage[width=0.6\paperwidth]{abstrusegoose}{img/ars_longa_vita_brevis.png}
\begin{frame}{Manualità (II)}
  \textbf{How to Teach Yourself Programming:}\footnote{\url{http://abstrusegoose.com/249}}
  \begin{center}
    \pgfuseimage{abstrusegoose}
  \end{center}
  
\end{frame}

\begin{frame}{Slides}

  Info sulle slide:
  \begin{itemize}
    \item le slide del corso saranno rese disponibili sul sito;
    \item segnalate pure eventuali errori;
    \item cercherò di pubblicare le slide in anticipo rispetto alla lezione;
    \item queste slide sono prodotte con \LaTeX \; \texttt{Beamer} (\emph{usate} \LaTeX!);
  \end{itemize}

  Segnalazioni di materiale:
  \begin{itemize}
    \item Materiale da voi prodotto;
    \item Cose interessanti che trovate online;
    \item Possiamo valutare insieme se riutilizzarle;
  \end{itemize}
\end{frame}