\subsection[Definizione ed esempi]{Definizione ed esempi}

\begin{frame}{Variabili (I)}

  Variabile\footnote{da \url{https://en.wikipedia.org/wiki/Variable_(computer_science)}}:
  \begin{quote}
    «In computer programming, a \textbf{variable} or scalar is a \alert<2->{storage location}
     paired with an \alert<3->{associated symbolic name (an identifier)}, which contains some
     known or unknown quantity of information referred to as a \alert<4->{value}.»
  \end{quote}

\end{frame}

\begin{frame}{Variabili (II)}

  Esempi:   
    \begin{columns}[T] % contents are top vertically aligned
      \begin{column}[T]{5cm} % each column can also be its own environment
	\begin{enumerate}[<+->]
	  \item un intero:
	  \begin{itemize}
	    \item \texttt{\textbf{int} i = 0;}
	    \item Pseudocodice $i \gets 0$
	  \end{itemize}
	  \item un numero con la virgola:
	  \begin{itemize}
	    \item \texttt{\textbf{double} pi = 3.14;}
	    \item (usate \texttt{Math.PI} per $\pi$)
	  \end{itemize}
	  \item un (singolo) carattere:
	  \begin{itemize}
	    \item \texttt{\textbf{char} k = \structure{'c'};}
	  \end{itemize}
	\end{enumerate}
      \end{column}
      \begin{column}[T]{7cm} % alternative top-align that's better for graphics
	\begin{enumerate}[<+->]
	  \setcounter{enumi}{3}
	  \item un \emph{valore booleano}:
	  \begin{itemize}
	    \item \texttt{\textbf{bool} val1 = \textbf{true};}
	    \item \texttt{\textbf{bool} val2 = \textbf{false};}
	  \end{itemize}
	  \item una \emph{stringa} di caratteri:
	  \begin{itemize}
	    \item \texttt{\textbf{String} yoda = \structure{"There is no try!"};}
	  \end{itemize}
	\end{enumerate}
      \end{column}
    \end{columns}
\end{frame}

\begin{frame}{Variabili (III)}

  Una variable:
  \begin{itemize}
    \item è un ``\emph{contenitore}'' di informazioni (= un certo numero di bytes allocati nella
	  memoria volatile del computer (RAM));
    \item è contraddistinta da un identificatore, negli esempi di prima \texttt{i}, \texttt{pi},
	  \texttt{yoda}, $\dots$
	  (\emph{case sensitive} \texttt{pippo} $\neq$ \texttt{Pippo} $\neq$ \texttt{PIPPO});
    \item nei linguaggi fortemente tipizzati hanno un \textbf{tipo}, negli esempi di prima
	  \texttt{\textbf{int}}, \texttt{\textbf{double}}, \texttt{\textbf{String}}, $\dots$;
  \end{itemize}

\end{frame}

\subsection[Dichiarazione e assegnamento]{Dichiarazione e assegnamento}

\begin{frame}{Dichiarazione e assegnamento}

  Le variabili possono essere create con la \textbf{dichiarazione}:
  \begin{itemize}
    \item \textbf{\structure{dichiarazione}}: \texttt{\textbf{int} i;} (\emph{specifica} il tipo di un identificatore);
    \item \textbf{\structure{assegnamento}}:  \texttt{i = 1;} (\emph{assegna} un valore a un identificatore);
    \item \textbf{\structure{inizializzazione}}: primo assegnamento \texttt{\textbf{int} i = 1;} (dichiarazione + inizializzazione);
  \end{itemize}

\end{frame}
