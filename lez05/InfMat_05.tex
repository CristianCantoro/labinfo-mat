\documentclass[10pt]{beamer}

% %%% Style %%%
\mode<presentation>
{
  \usetheme{Berlin}
  \setbeamertemplate{blocks}[rounded]
  \setbeamercolor{block title}{bg=gray}
  \setbeamercovered{transparent}
}

% %%% Packages %%%
% Babel and fonts
\usepackage[english]{babel}
\usepackage[utf8]{inputenc}

% Graphics and images
\usepackage{xcolor}
\usepackage{graphicx}
\DeclareGraphicsExtensions{.pdf,.png,.jpg, .eps}
\usepackage{relsize}
\usepackage{colortbl}


% Math notation and symbols
\usepackage{mathrsfs}
\usepackage{amsthm}
\usepackage{bbold}
\usepackage{amsfonts}
\usepackage{amsmath}
\usepackage{amssymb}
% \usepackage{algorithm}
\usepackage{algpseudocode}
\usepackage{listings}

% \DeclareMathOperator*{\len}{\text{length of }}
% Backup slides
\usepackage{appendixnumberbeamer}

% Tables, Colums and the like
\usepackage{longtable}
\usepackage{listings}

% Hyperref
\usepackage{hyperref} 

% Boxes
\usepackage{fancybox}
\usepackage{lmodern}
\usepackage{tikz}
\usepackage{tcolorbox}
\usepackage{mdframed}	

\usepackage{hyperref}
\usepackage{listings}
\usepackage{setspace}
\usepackage{subfiles}
\usepackage{alltt}
\usepackage{mathtools}
\usepackage{fancyvrb}
\usepackage{url}

\usepackage{tikz}
\usetikzlibrary{shapes,arrows,positioning,calc}

\usepackage{xparse}
\usepackage{ifthen}
\usepackage{twoopt}

% Custom packages (needs absolute path from root document
%  Usage: \usepackage{appendixnumberbeamer}
%  Custom title page

\makeatletter

\setbeamertemplate{footline}
{
  \leavevmode%
  \hbox{%
  \begin{beamercolorbox}[wd=0.25\paperwidth,ht=2.25ex,dp=1ex,center]{author in head/foot}%
    \usebeamerfont{author in head/foot}\insertshortauthor
  \end{beamercolorbox}%
  \begin{beamercolorbox}[wd=0.50\paperwidth,ht=2.25ex,dp=1ex,center]{title in head/foot}%
    \usebeamerfont{title in head/foot}\inserttitle
  \end{beamercolorbox}%
  \begin{beamercolorbox}[wd=0.25\paperwidth,ht=2.25ex,dp=1ex,right]{date in head/foot}%
    \insertframenumber{} / \inserttotalframenumber\hspace*{2ex} 
  \end{beamercolorbox}}%
  \vskip0pt%
}
\makeatother

\makeatletter
\setbeamertemplate{headline}
{
  \leavevmode%
  \hbox{%
  \begin{beamercolorbox}[wd=\paperwidth,ht=2.25ex,dp=1ex,center]{title in head/foot}%
  \end{beamercolorbox}%
  }%
  \vskip0pt%
}
\makeatother

\defbeamertemplate*{title page}{customized}[1][]
{
  \begin{center}
    \begin{variableblock}{}{}{bg=blue}
      \begin{center}
	\usebeamerfont{title}\inserttitle\par
      \end{center}
    \end{variableblock}
  \end{center}

  \begin{center}
    \begin{minipage}{1.0\textwidth}
      \begin{center}
	\insertauthor

	\begin{scriptsize}
	  \insertdate
	\end{scriptsize}

      \end{center}
    \end{minipage}
  \end{center}
}

\input{includes/appendixnumberbeamer.sty}

% %%% Macros and custom commands %%%
\DeclareMathOperator*{\len}{\textbf{length of }}

\newcommand{\btVFill}{\vskip0pt plus 1filll}

\setbeamercolor{structure}{fg=cyan!90!black}

\newenvironment{variableblock}[3]{%
\setbeamercolor{block body}{#2}
\setbeamercolor{block title}{#3}
\begin{block}{#1}}{\end{block}}

\lstdefinestyle{JavaPlain}{ %
basicstyle=\scriptsize\ttfamily, % the size of the fonts 
numbers=left,                   % where to put the line-numbers
numberstyle=\tiny,      % the size of the fonts that are used for th
stepnumber=1,                   % the step between two line-numbers
numbersep=5pt,                  % how far the line-numbers are from the code
backgroundcolor=\color{white},  % choose the background color
showspaces=false,               % show spaces adding particular underscores
showstringspaces=false,         % underline spaces within strings
showtabs=false,                 % show tabs within strings adding 
frame=single,           % adds a frame around the code
tabsize=2,          % sets default tabsize to 2 spaces
captionpos=b,           % sets the caption-position to bottom
breaklines=true,        % sets automatic line breaking
breakatwhitespace=false,    % sets if automatic breaks should only happen
fancyvrb=true,
fvcmdparams=textbf 1 textit 1,
}

% Colors
\definecolor{ballblue}{rgb}{0.13, 0.67, 0.8}
\definecolor{brown(web)}{rgb}{0.65, 0.16, 0.16}
\definecolor{brown(traditional)}{rgb}{0.59, 0.29, 0.0}
\definecolor{amber}{rgb}{1.0, 0.75, 0.0}

\newcommand\Red[1]{\textcolor{red}{#1}}
\newcommand\Green[1]{\textcolor{green!50!black}{#1}}
\newcommand\LightGreen[1]{\textcolor{green!60!black}{#1}}
\newcommand\Blue[1]{\textcolor{blue!60!white}{#1}}
\newcommand\Violet[1]{\textcolor{violet}{#1}}
\newcommand\LightBlue[1]{\textcolor{ballblue}{#1}}
\newcommand\DarkBlue[1]{\textcolor{blue!60!black}{#1}}
\newcommand\Grey[1]{\textcolor{gray}{#1}}
\newcommand\Gray[1]{\textcolor{gray}{#1}}
\newcommand\Brown[1]{\textcolor{brown(web)}{#1}}
\newcommand\LightBrown[1]{\textcolor{brown(traditional)}{#1}}
\newcommand\Yellow[1]{\textcolor{amber}{#1}}

\newenvironment{redtext}{\color{red}}{\ignorespacesafterend}

% Helpers
\newcommand\Jcomment[1]{\LightGreen{// #1}}
\newcommand\JcommentMulti[1]{\LightGreen{/* #1}}
\newcommand\String[1]{\textcolor{blue!80!blue}{#1}}
\newcommand\Word[1]{\textcolor{purple!90!red}{#1}}
\newcommand\bang{!}
\newcommand\pipe{\|}

% Prints
% \newcommand\Jprintf[2][]{\LightBlue{printf}(\String{#1}, #2)}
\newcommandtwoopt{\Jprintf}[2][-NoValue-][-NoValue-]{%
    \ifthenelse{\equal{#2}{-NoValue-}}{\LightBlue{printf}(\String{#1})}{\LightBlue{printf}(\String{#1}, #2)}%
}

\newcommand\Jprintln[1]{\LightBlue{println}(\String{#1})}
\newcommand\JprintLN{\LightBlue{println}}

\newcommand\Jprint[1]{\LightBlue{print}(\String{#1})}
\newcommand\JPrint{\LightBlue{print}}

% Control structures
% Shortcut for JavaFor:
% \JavaFor (\Blue!init| \Green!n <= Math.sqrt(numero);| \Violet!n++|)
\newcommand\JavaForColors[3]{\Blue{#1}; \Green{#2}; \Violet{#3}}

\newcommand\Jfor[1][-NoValue-]{%
  \ifthenelse{\equal{#1}{-NoValue-}}{\Word{for}}{\JavaForO[#1]}%
}

\newcommandtwoopt{\JavaForO}[3][-NoValue-][-NoValue-]{%
    \JavaForOptions{#1}{#2}{#3}%
}

\DeclareDocumentCommand\JavaForOptions{ ggg }{%
  \Word{for}\IfValueT{#1}{%
    (\JavaForColors{#1}{#2}{#3}) %
  }%
}

% Shortcut for JavaFor:
% \JavaFor (\Blue!init| \Green!n <= Math.sqrt(numero);| \Violet!n++|)
\newcommand\Jif[1][-NoValue-]{%
  \ifthenelse{\equal{#1}{-NoValue-}}{\Word{if}}{\Word{if} (\Green{#1})}%
}

% Shortcut for JavaWhile:
% \JavaWhile(\Blue!init| \Green!n <= Math.sqrt(numero);| \Violet!n++|)
\newcommand\Jwhile[1][-NoValue-]{%
  \ifthenelse{\equal{#1}{-NoValue-}}{\Word{while}}{\Word{while} (\Green{#1})}%
}

% Other reserved words
\newcommand\Jthis{\textbf{\Word{this}}}
\newcommand\Jstatic{\Word{static}}
\newcommand\Jpublic{\Word{public}}
\newcommand\Jprivate{\Word{private}}
\newcommand\Jprotected{\Word{protected}}
\newcommand\Jdouble{\Word{double}}
\newcommand\Jfloat{\Word{float}}
\newcommand\Jint{\Word{int}}
\newcommand\Jvoid{\Word{void}}
\newcommand\Jreturn{\Word{return}}
\newcommand\Jclass{\Word{class}}
\newcommand\Jargs{\LightBrown{args}}
\newcommand\Jimport{\Word{import}}
\newcommand\Jnew{\Word{new}}
\newcommand\JsystemIn{System.\DarkBlue{\textit{\textbf{in}}}}
\newcommand\Jthrows{\Word{throws}}
\newcommand\Jtry{\Word{try}}
\newcommand\Jcatch{\Word{catch}}

\newenvironment{JavaCodePlain}[1][]
  { \VerbatimEnvironment%
    \begin{Verbatim}[#1]}
  { \end{Verbatim}  } 

\algdef{SE}[DOWHILE]{Do}{DoWhile}{\algorithmicdo}[1]{\algorithmicwhile\ #1}%
\renewcommand{\algorithmicrequire}{\textbf{Input:}}
\renewcommand{\algorithmicensure}{\textbf{Output:}}

\title[Laboratorio di Informatica - Lezione 5]{Laboratorio di Informatica \\ Lezione 5}
\author[Cristian Consonni]{Cristian Consonni}
\date[28/10/2015]{11 novembre 2015}
\institute[UniTN]{Università degli Studi di Trento}

% %%% Put slides with section name at the begining of each section %%%
\AtBeginSection[]
{
  \begin{frame}<beamer>
    \frametitle{Outline for section \thesection}
    \tableofcontents[currentsection]
  \end{frame}
}
\setbeamertemplate{navigation symbols}{}

\begin{document}

%%%%%%%%%% TITLE %%%%%%%%%%
% Formerly part of title.tex
\begin{frame}
  \titlepage
\end{frame}

\begin{frame}{Outline}
  \tableofcontents
\end{frame}
%%%%%%%%%% END TITLE %%%%%%%%%%

\begin{frame}{Chi sono}
  Cristian Consonni

  \begin{itemize}
    \item \textbf{DISI - Dipartimento di Ingegneria e Scienza dell'Informazione}
    \item \textbf{Pagina web} del laboratorio: \structure{\url{http://disi.unitn.it/~consonni/teaching}}
    \item \textbf{Email}: \structure{\url{cristian.consonni@unitn.it}}
    \item \textbf{Ufficio}: Povo 2 - Open Space 9
      \begin{itemize}
	\item Per domande: scrivetemi una mail
	\item Ricevimento: su appuntamento via mail
      \end{itemize}
  \end{itemize}
\end{frame}


\section{Introduzione alla Programmazione Orientata agli Oggetti (OOP)}
\begin{frame}{Programmazione Orientata agli Oggetti (I)}
  
  \begin{itemize}
   \item La Programmazione Orientata agli Oggetti (o \emph{Object Oriented Programming}, OOP, in inglese)
         è un paradigma di Programmazione basato sui concetti di \textbf{oggetto} e \textbf{classe}.
   \item In questa lezione introdurremo i concetti di base legati agli oggetti e li utilizzeremo
         come una \textbf{struttura dati}, ovvero una entità che permette di gestire un insieme di
         dati (detti \emph{attributi} o \emph{membri}).
  \end{itemize}

\end{frame}

\begin{frame}{Programmazione Orientata agli Oggetti (II)}
  
  Finora \textbf{\Red{ad eccezione delle \texttt{String}}} abbiamo usato delle variabili di uno dei
  tipi \textbf{primitivi} (o \emph{atomici}):
  \begin{itemize}
    \item \texttt{\Word{int}}
    \item \texttt{\Word{double}}
    \item \texttt{\Word{float}}
    \item \dots
  \end{itemize}
  

  utilizzando un tipo primitivo sappiamo che i dati verrano rappresentati in memeria in un 
  un modo predefinito:
  \begin{itemize}
   \item \texttt{\Word{int}} $\rightarrow$ 4 byte (32 bit) 
   \begin{itemize}
    \item min: $-2^{31}$;
    \item max: $2^{31} - 1$;
   \end{itemize}

   \item \texttt{\Word{double}} $\rightarrow$ 8 byte (64 bit)
   \begin{itemize}
    \item $max \equiv |min|$: $1.7976931348623157 \cdot 10^{308}$;
    \item $\varepsilon$: $4.9 \cdot 10^{-324}$;
   \end{itemize}
  \end{itemize}

\end{frame}

\begin{frame}[fragile]\frametitle{Programmazione Orientata agli Oggetti (III)}
  
  Immaginiamo di volere scrivere un programma che memorizza i dati (nome e cognome,
  luogo di nascita, età) di una persona.
  \begin{JavaCodePlain}[commandchars=\\!|]
  
  String nome1 = \String!"Alice Rossi"|;
  String luogo1 = \String!"Milano"|;
  \Jint eta1 = 19;
  
  String nome2 = \String!"Roberto Verdi"|;
  String cognome2 = \String!"Roma"|;
  \Jint  eta2 = 20;
  \dots

  \end{JavaCodePlain}
  
  Problemi di questo approccio:
  \begin{itemize}
   \item   È molto scomodo creare nuove istanze dell'entità (concettuale) ``Persona'';
   \item   È molto facile commettere errori;
  \end{itemize}

\end{frame}

\begin{frame}[fragile]\frametitle{Programmazione Orientata agli Oggetti (IV)}

  Una \textbf{oggetto}:
  \begin{itemize}
    \item è un'\emph{entità software} costituito da un insieme di stati e comportamenti che si riferiscono a uno stesso concetto; \\
          {\footnotesize (\emph{«An object is a software bundle of related state and behavior.»}) }
          {\scriptsize \LightBlue{\url{https://docs.oracle.com/javase/tutorial/java/concepts/}} }
    \item consiste in una regione di memoria allocata;

  \end{itemize}
\end{frame}

\begin{frame}[fragile]\frametitle{Programmazione Orientata agli Oggetti (V)}

  Una \textbf{classe}:
  \begin{itemize}
    \item è un prototipo a partire dalla quale gli oggetti vengono \textbf{creati} (o \textbf{istanziati}, 
	  o \textbf{instanziati});\\
          {\footnotesize (\emph{«A class is a blueprint or prototype from which objects are created.»}) }
          {\scriptsize \LightBlue{\url{https://docs.oracle.com/javase/tutorial/java/concepts/}} }
    \item è la specifica che descrive quali sono i possibili stati e comportamenti di un oggetto creati
          a partire da esso;
    \item è un insieme di variabili (anche dette \textbf{\emph{membri}} o \textbf{\emph{attributi}}) e di
         funzioni i metodi (dette \textbf{\emph{metodi}});
    \item è anche chiamata \emph{Abstract Data Type} (ADT).
  \end{itemize}
\end{frame}


\pgfdeclareimage[width=0.35\paperwidth]{people01}{img/people_jumping_01.png}
\pgfdeclareimage[width=0.35\paperwidth]{people02}{img/people_jumping_02.png}
\pgfdeclareimage[width=0.35\paperwidth]{people03}{img/people_jumping_03.png}

\begin{frame}[fragile]\frametitle{Programmazione Orientata agli Oggetti (VI)}


  \begin{columns}[T]
    \begin{column}[T]{7cm}
    Classe:
    \begin{JavaCodePlain}[commandchars=\\!|]
    
    \Jpublic \Jclass Persona {

      \Jprivate String nome;      
      \Jprivate String luogoNascita;
      \Jprivate \Jint eta;

    }

    \end{JavaCodePlain}
    \end{column}

    \begin{column}[T]{5cm}
    Oggetti:
    \begin{center}
      \pgfuseimage{people01}
    \end{center}
    \end{column}
  \end{columns}


  \begin{minipage}[b]{12cm}
  {\scriptsize Photo credit: CC-BY-SA Mark Sebastian @ Wikimedia Commons \url{http://bit.ly/1kip83N}}
  \end{minipage}
\end{frame}

\begin{frame}[fragile]\frametitle{Programmazione Orientata agli Oggetti (VI)}


  \begin{columns}[T]
    \begin{column}[T]{7cm}
    Classe:
    \begin{JavaCodePlain}[commandchars=\\!|]
    
    \Jpublic \Jclass Persona {

      \Jprivate String nome;      
      \Jprivate String luogoNascita;
      \Jprivate \Jint eta;

    }
    
    \dots
    
    Persona Alice;

    \end{JavaCodePlain}
    \end{column}

    \begin{column}[T]{5cm}
    Oggetti:
    \begin{center}
      \pgfuseimage{people02}
    \end{center}
    \end{column}
  \end{columns}

\end{frame}

\begin{frame}[fragile]\frametitle{Programmazione Orientata agli Oggetti (VI)}


  \begin{columns}[T]
    \begin{column}[T]{7cm}
    Classe:
    \begin{JavaCodePlain}[commandchars=\\!|]
    
    \Jpublic \Jclass Persona {

      \Jprivate String nome;      
      \Jprivate String luogoNascita;
      \Jprivate \Jint eta;

    }

    \dots
    
    Persona Alice;
    Persona John;

    \end{JavaCodePlain}
    \end{column}

    \begin{column}[T]{5cm}
    Oggetti:
    \begin{center}
      \pgfuseimage{people03}
    \end{center}
    \end{column}
  \end{columns}
  
\end{frame}

\begin{frame}[fragile]\frametitle{Programmazione Orientata agli Oggetti (VI)}


  \begin{columns}[T]
    \begin{column}[T]{7cm}
    Classe:
    \begin{JavaCodePlain}[commandchars=\\!|]
    
    \Jpublic \Jclass Persona {

      \Jprivate String nome;      
      \Jprivate String luogoNascita;
      \Jprivate \Jint eta;

    }

    \dots
    
    Persona Alice;
    Persona John;

    \end{JavaCodePlain}
    \end{column}

    \begin{column}[T]{5cm}
    Oggetti:
    \begin{center}
      \pgfuseimage{people03}
    \end{center}
    \end{column}
  \end{columns}

  \begin{alertblock}{Oggetti vs. Classi}
    Un oggetto è un'istanza di una classe
  \end{alertblock}
  
\end{frame}

\begin{frame}[fragile]\frametitle{Classi vs. Oggetti (I)}

  \begin{columns}
    \begin{column}{6cm}
      Classe:
      \begin{itemize}
	\item descrizione delle proprietà \emph{comuni} a una tipologia di variabili;
	\item \`e un ``concetto'';
	\item \`e una parte di un programma;
      \end{itemize}
      
%       \begin{enumerate}
%        \item Persona
%        \item Album
%        \item Canzoni
%       \end{enumerate}
    \end{column}

    \begin{column}{6cm}
      Oggetti:
      \begin{itemize}
	\item rappresentazioni delle proprietà di una singola istanza;
	\item Un ``fenomeno''/``manifestazione''.
	\item una parte dei dati nell'esecuzione di un programma.
      \end{itemize}
      
%       \begin{enumerate}
%        \item Hillary Clinton, Rafael Nadal, Lewis Hamilton;
%        \item Thriller, Back in Black, The Dark Side of the Moon
%        \item {\scriptsize Thriller, Beat it, Billie Jean, Hells Bells, Shoot to Thrill, Back in Black, \dots}
%       \end{enumerate}
    \end{column}
  \end{columns}

  ${}$
  \begin{columns}[T]
    \begin{column}[T]{6cm}
      \begin{enumerate}
       \item Persona
       \item Album
       \item Canzoni
      \end{enumerate}
    \end{column}

    \begin{column}[T]{6cm}    
      \begin{enumerate}
       \item Hillary Clinton, Rafael Nadal, Lewis Hamilton;
       \item Thriller, Back in Black, The Dark Side of the Moon
       \item {\small Thriller, Beat it, Billie Jean, Hells Bells, Shoot to Thrill, Back in Black, \dots}
      \end{enumerate}
    \end{column}
  \end{columns}

\end{frame}

\pgfdeclareimage[width=0.25\paperwidth]{people01}{img/muffins.jpg}
\begin{frame}[fragile]\frametitle{Classi vs. Oggetti (II)}

  \begin{itemize}
   \item Nella programmazione ad oggetti (OOP) si scrivono classi;
    \begin{itemize}
      \item il codice sorgente che scriviamo contiene delle \textbf{classi};
      \item una classe è ``statica'';
      \item ``Una'';
    \end{itemize}
   \end{itemize}

   \begin{itemize}
   \item Gli oggetti sono creati a partire dalle classi;
    \begin{itemize}
      \item una classe è come una ricetta di una torta, gli oggetti sono le torte prodotte
      a partire dalla ricetta.
      \item un oggetto è ``dinamico''
      \item ``Molti''
    \end{itemize}
  \end{itemize}

  ${}$
  \begin{columns}[T]
    \begin{column}[T]{4cm}
      Ricetta (Classe) \textbf{Muffin}:
      \begin{enumerate}
       \item 3 uova;
       \item 380 g di farina;
       \item \dots
      \end{enumerate}
    \end{column}

    \begin{column}[T]{7cm}    
     \pgfuseimage{people01}
    \end{column}
  \end{columns}
  ${}$
  {\scriptsize Photo credit: CC-BY Sara K @ Flickr \url{http://bit.ly/1kioQdd}}
\end{frame}

\section{Oggetti e Classi}

\begin{frame}[fragile]\frametitle{Classi: costruttore (I)}

  Una classe necessita di un \textbf{costruttore} per potere essere istanziata:
  
  \begin{JavaCodePlain}[commandchars=\\!|]
  \Jpublic \Jclass Persona {
    \Jcomment!attributi della classe -> variabili dell'oggetto|
    \Jprivate String nome;      
    \Jprivate String luogoNascita;
    \Jprivate \Jint eta;

    \Jcomment!Costruttore con parametri|
    \Jpublic Persona(String nome, String luogoNascita, \Jint eta) {
      \dots
    }

    \Jcomment!Costruttore senza parametri con valori di default|
    \Jpublic Persona() {
      \dots
    }    
  }
  \end{JavaCodePlain}

\end{frame}

\begin{frame}[fragile]\frametitle{Classi: costruttore (II)}

  Il \textbf{costruttore} è il \emph{metodo} che crea la classe:
  
  \begin{JavaCodePlain}[commandchars=\\!|]
  \Jpublic \Jclass Persona {
    \dots
    \Jcomment!Costruttore con parametri|
    \Jpublic Persona(String nome, String luogoNascita, \Jint eta) {
      \Jthis.nome = nome;
      \Jthis.luogoNascita = luogo;
      \Jthis.eta = eta;
    }

    \Jcomment!Costruttore senza parametri con valori di default|
    \Jpublic Persona() {
      \Jthis.nome = \String!"Anonimo"|;
      \Jthis.luogoNascita = \String!"Sconosciuto"|;
      \Jthis.eta = -1;
    }
  }
  \end{JavaCodePlain}

\end{frame}

\begin{frame}[fragile]\frametitle{Classi: costruttore (III)}

  La parola riservata \Jthis{} serve per specificare se ci si sta riferendo
  ai parametri del costruttore o ai membri della classe.
  
  \begin{JavaCodePlain}[commandchars=\\!|]
  \dots
    \Jpublic Persona(String nome, String luogoNascita, \Jint eta) {
      \Jthis.nome = nome;
      \Jthis.luogoNascita = luogo;
      \Jthis.eta = eta;
  \dots
  \end{JavaCodePlain}

\end{frame}

\begin{frame}[fragile]\frametitle{La parola riservata \texttt{this}}
  
  \begin{itemize}
   \item Costruttore con parametri:
  \end{itemize}
  \begin{JavaCodePlain}[commandchars=\\!|]
    \Jpublic Persona(String nome, String luogoNascita, \Jint eta) {
      \Jthis.nome = nome;
      \Jthis.luogoNascita = luogo;
      \Jthis.eta = eta;
  
  Persona anna = Persona(\String!"Anna Rossi"|, \String!"Milano"|, 19);
  \end{JavaCodePlain}

  \begin{itemize}
   \item Costruttore senza parametri:
  \end{itemize}
  \begin{JavaCodePlain}[commandchars=\\!|]
    \Jpublic Persona() {
      \Jthis.nome = \String!"Anonimo"|;
      \Jthis.luogoNascita = \String!"Sconosciuto"|;
      \Jthis.eta = -1;
    }
  
  Persona anon = Persona();
  \end{JavaCodePlain}

\end{frame}

\begin{frame}{Accessibilità delle variabili (I)}

  Le parole riservate \Jprivate, \Jpublic, \Jprotected{} modificano l'accessibilità delle variabili:
  \begin{enumerate}
   \item \Jpublic: sono accessibili da chiunque. (\Green{\textbf{più accessibili}})
   \item \Jprotected: sono accessibili solo all'interno dallo stesso package dalla stessa classe e dalle sottoclassi.
   \item default (nessun modificatore aggiuntivo): sono accessibili solo all'interno dallo stesso package dalla stessa classe.
   \item \Jprivate: sono accessibili sollo dalla stessa classe. (\Red{\textbf{meno accessibili}})
  \end{enumerate} 

  \pause{
  \begin{alertblock}{public e private}
    Noi useremo solo \textbf{public} (\Green{\textbf{più accessibili}}) e \textbf{private} (\Red{\textbf{meno accessibili}})
  \end{alertblock}
  }

\end{frame}

\begin{frame}{Accessibilità delle variabili (II)}

%   \begin{table}[]
%   \centering
%     \begin{tabular}{l|llll}
%       Modificatore                                                        & \multicolumn{1}{c}{Classe} & \multicolumn{1}{c}{Package} & \multicolumn{1}{c}{Sottoclasse} & \multicolumn{1}{c}{\begin{tabular}[c]{@{}c@{}}Resto del\\ mondo\end{tabular}} \\ \hline
%       \begin{tabular}[c]{@{}l@{}}\textbf{public}\\${}$\end{tabular}           & Sì                         & Sì                          & Sì                              & Sì                                                                            \\
%       \begin{tabular}[c]{@{}l@{}}protected\\${}$\end{tabular}                 & Sì                         & Sì                          & Sì                              & No                                                                            \\
%       \begin{tabular}[c]{@{}l@{}}default\\ (no modificatore)\end{tabular} & Sì                         & Sì                          & No                              & No                                                                            \\
%       \begin{tabular}[c]{@{}l@{}}\textbf{private}\\${}$\end{tabular}          & Sì                         & No                          & No                              & No                                                                            
%     \end{tabular}
%   \end{table}  

  La tabella seguente riassume l'accessibilità delle variabili dichiarate con i vari modificatori.
  \begin{table}[htbp]
    \begin{tabular}{|l|cccc|}
      \hline
      \multicolumn{1}{|c|}{\textbf{Modificatore}} & \textbf{Class} & \textbf{Package} & \textbf{Sottoclassi} & \textbf{Mondo} \\ \hline
      \rowcolor[HTML]{9AFF99}
      \textbf{public} & \textbf{Y} & \textbf{Y} & \textbf{Y} & \textbf{Y} \\ \hline
      protected & Y & Y & Y & N \\ \hline
      default
      (no modificatore) & Y & Y & N & N \\ \hline
      \rowcolor[HTML]{FD6864}
      \textbf{private} & \textbf{Y} & \textbf{N} & \textbf{N} & \textbf{N} \\ \hline
    \end{tabular}
  \end{table}
  {\footnotesize Si veda anche:}\\
  {\scriptsize \LightBlue{\url{https://docs.oracle.com/javase/tutorial/java/javaOO/accesscontrol.html}} }

\end{frame}

\begin{frame}{Accessibilità delle variabili: metodi getter e setter (I)}

  Dato che le variabili \Jprivate{} non sono accessibili al di fuori della classe è necessario creare dei metodi
  che restituiscono e modificano il loro valore. Questi metodi sono detti rispettivamente:
  \begin{itemize}
   \item \textbf{get}ter: restitituisce (\textbf{get}) il valore della variabile, ad es. \texttt{getNome()};
   \item \textbf{set}ter: imposta (\textbf{set}) il valore della variabile, ad es. \texttt{setNome()};
  \end{itemize}
  di solito i nome sono \texttt{getNomeVariabile()} e \texttt{setNomeVariabile()};
   
  \pause{
  \begin{alertblock}{Variabili private vs. public}
    Se una variabile è \Jpublic{} non è necessario avere getter e setter, ma è buona norma usare 
    variabili \Jprivate{} nelle classi
  \end{alertblock}
  }

\end{frame}

\begin{frame}[fragile]\frametitle{Accessibilità delle variabili: metodi getter e setter (II)}

  Esempio di metodi \textbf{get}ter e \textbf{set}ter per la variabile \texttt{nome} nella classe
  Persona:
  
  \begin{JavaCodePlain}[commandchars=\\!|]
  \Jpublic \Jclass Persona {

    \Jprivate String nome;
    \dots
    
    \Jpublic String getNome() {
      \Jreturn nome;
    }

    \Jpublic \Jvoid setNome(String nome) {
      \Jthis.nome = nome;
    }

  }
  \end{JavaCodePlain}

\end{frame}

\begin{frame}{Metodo \texttt{toString} (I)}

  \begin{itemize}
    \item il metodo \texttt{toString()} restituisce una \textbf{stringa} ovvero la ``rappresentazione testuale''
	  dell'oggetto su cui è invocato (da usare ad esempio quando si stampa l'oggetto). 
    \item la definizione default del metodo nella classe Object è \Blue{\texttt{$<nome\_classe>@<hashcode>$}};
    \item possiamo ridefinire questo metodo per stampare un oggetto nel modo che vogliamo;
  \end{itemize}

\end{frame}

\begin{frame}[fragile]\frametitle{Metodo \texttt{toString} (II)}

  Esempio:
  \begin{JavaCodePlain}[commandchars=\\!|]
  \Jpublic \Jclass Persona {
  
    \Jprivate String nome;
    \Jprivate String luogoNascita;
    \Jprivate \Jint eta;

    \dots

    \Jpublic String toString() {
      \Jreturn nome + \String!" ("| + luogoNascita + \String!"), "| + eta;
    }
  }
  \end{JavaCodePlain}

  \begin{itemize}
   \item[$\Rightarrow$] \JprintLN(anna); $\Longrightarrow$ \texttt{Anna Rossi (Milano), 19}
  \end{itemize}

\end{frame}

\begin{frame}{Accessibilità delle classi}

  A livello delle classe \`e possibile usare:
  \begin{enumerate}
   \item \Jpublic: classe accessibile anche da altri package.
   \item default (nessun modificatore aggiuntivo): classe accessibile solo all'interno dello stesso package.
  \end{enumerate}

\end{frame}

\begin{frame}{Struttura di un progetto (I)}

  In un progetto esistono più classi:
  \begin{itemize}
   \item Una classe principale che contiene il metodo \textbf{main};
   \item altri file contenenti le altre classi ``secondarie'': create un nuovo file per ogni classe
   (ad esempio \texttt{Persona.java} per la classe \texttt{Persona});
   \item il nome di una classe ha l'iniziale maiuscola e si usa la notazione CamelCase/PascalCase;
   \item non si possono avere due classi con lo stesso nome all'interno dello stesso progetto;
  \end{itemize}
  
\end{frame}

\begin{frame}[fragile]\frametitle{Struttura di un progetto (II)}

  Esempio:
  \begin{itemize}
   \item creiamo un progetto EsempioProgetto che riguarderà la musica (album e canzoni);
   \item La classe principale si chiama \texttt{EsempioProgetto} e si trova nel file \texttt{EsempioProgetto.java};
   \item Esistono due classi \texttt{Album} (\texttt{Album.java}) e \texttt{Canzone} (\texttt{Canzone.java}) che possono
         essere usate dentro \texttt{EsempioProgetto}.
  \end{itemize}

\end{frame}

\pgfdeclareimage[width=0.35\paperwidth]{progetto}{img/progetto.png}
\begin{frame}[fragile]\frametitle{Struttura di un progetto (II)}

  \begin{center}
    \pgfuseimage{progetto}
  \end{center}

  Classe:
  \begin{JavaCodePlain}[commandchars=\\!|]
  
  \Jpublic \Jclass EsempioProgetto {
  
    \Jpublic \Jstatic \Jvoid main(String[] \Jargs) {
    
      Album thriller = \Jnew Album(\String!"Thriller"|, \dots);
      Canzone hb = new Canzone(\String!"Hells Bells"|, \String!"AC/DC"|, \String!"5:13"|);
    }
  }
  \end{JavaCodePlain}

\end{frame}


\section{Utilizzo del debugger}

\begin{frame}{Bug}

  \begin{itemize}
   \item un \textbf{bug} (o \textbf{baco}) indica un errore o un comportamento inaspettato
   in un programma software;
   \item l'operazione di individuazione e correzione degli errori in un programma si chiama
   \textbf{debug}/\textbf{debugging};
   \item uno degli strumenti per facilitare l'operazione di debugging è il \textbf{debugging}
  \end{itemize}

\end{frame}

\pgfdeclareimage[width=0.25\paperwidth]{ada}{img/lovelace.jpg}
\begin{frame}{Bug (intermezzo storico) (I)}

  Piccola nota storica:
    \begin{columns}[T]
      \begin{column}{8cm}
	\begin{itemize}
	  \item in inglese \textbf{bug} significa \emph{insetto};
	  \item A livello teorico l'idea che un programma potesse contenere errori \`e stata avanzata
	  per la prima volta nel 1843 da \textbf{Ada Lovelace} nelle sue note sulla \textbf{macchina analitica} di \textbf{Babbage};
	  \item pare che l'espressione ``bug'' fosse in uso per indicare malfunzionamenti meccanici sin dal fine del 1800
	  (Thomas Edison, 1878)
	\end{itemize}

    \end{column}

    \begin{column}[T]{4cm}
	\begin{center}
	 \pgfuseimage{ada}
	\end{center}
    \end{column}
  \end{columns}


\end{frame}

\pgfdeclareimage[width=0.25\paperwidth]{bug}{img/bug.jpg}
\begin{frame}{Bug (intermezzo storico) (II)}

    \begin{columns}[T]
      \begin{column}{8cm}
	\begin{itemize}
    	  \item il primo bug informatico documentato della storia era un vero insetto (una falena),
	  che si era infilato in un rel\`e causando il malfunzionamento di un programma in un computer
	  all'università di Harvard, l'Harvard Mark II, il 9 settembre 1947. 
	  \item in seguito a quell'episodio il termine bug per indicare un problema informatico è diventato
	  comune grazie a \textbf{Grace Hopper} che era in capo al progetto Mark II.
	  (anche l'inventrice del primo compilatore).
	\end{itemize}
    \end{column}

    \begin{column}[T]{4cm}
	\begin{center}
	 \pgfuseimage{bug}
	\end{center}
    \end{column}
  \end{columns}

\end{frame}

\begin{frame}{Debugging (I)}

  \begin{itemize}
    \item durante il debugging un programma viene eseguito in modo interattivo per potere osservare
    l'esedcuzione del codice e il valore delle variabili passo dopo passo;
    \item si possono definire dei ``punti di controllo'', detti \textbf{breakpoints} e \textbf{watchpoints};
    \begin{itemize}
     \item nei \textbf{breakpoints} l'esecuzione di un programma viene interrotta per poterlo ispezionare;
     \item nei \textbf{watchpoints} l'esecuzione di un programma viene interrotta solo se una variabile
     viene letta o modificata;
    \end{itemize}

  \end{itemize}

\end{frame}

\pgfdeclareimage[width=0.5\paperwidth]{debug_breakpoint}{img/debug/debug_breakpoint.png}
\begin{frame}{Debugging (II)}

  \begin{itemize}
   \item Eclipse permette di lanciare un programma in \textbf{debug mode} e di usare una \textbf{debug perspective}
   allo scopo di eseguire il debugging;
   \item è possibile inserire un breakpoint cliccando all'inizio della riga con il tasto destro e selezionando
   \textbf{Toggle Breakpoint};
  \end{itemize}
  
  \begin{center}
   \pgfuseimage{debug_breakpoint}
  \end{center}

  \begin{itemize}
   \item Per lanciare il programma premere con il tasto destro sul nome del file e selezionare
   \textbf{Debug As $>$ Java Application};
  \end{itemize}

\end{frame}

% \pgfdeclareimage[width=0.25\paperwidth]{debug01}{img/debug/debug01.png}
{
  \setbeamercolor{background canvas}{bg=}
  \includepdf[pages={1}]{img/debug/debug01.pdf}
}

\pgfdeclareimage[width=0.5\paperwidth]{debug_perspective}{img/debug/debug_perspective.png}
\begin{frame}{Debugging (V)}

  Confermare l'apertura della \emph{debug perspective}:
  \begin{center}
   \pgfuseimage{debug_perspective}
  \end{center}

\end{frame}

% \pgfdeclareimage[width=0.25\paperwidth]{debug01}{img/debug/debug01.png}
{
  \setbeamercolor{background canvas}{bg=}
  \includepdf[pages={1}]{img/debug/debug02.pdf}
}

{
  \setbeamercolor{background canvas}{bg=}
  \includepdf[pages={1}]{img/debug/debug03.pdf}
}

{
  \setbeamercolor{background canvas}{bg=}
  \includepdf[pages={1}]{img/debug/debug04.pdf}
}

\begin{frame}{Debugging (IV)}

  Il debugger pu\`o essere comandato attraverso le icone nella barra o i tasti funzione:
  \begin{itemize}
    \item \textbf{F5} esegue la linea selezionata e va alla riga successiva. Se la linea selezionata \`e una chiamata a funzione allora il debugger entra nella funzione indicata;
    \item \textbf{F6} ``step over'': esegue un metodo senza entrare esplicitamente in esso con il debugger;
    \item \textbf{F7} ``step out'': ritorna alla funzione che ha chiamato il metodo corrent, ovvero termina l'esecuzione del metodo corrente e ritorna al chiamante;
    \item \textbf{F8} riprende l'esecuzione del programma fino a che non viene incontrato un nuovo breakpoint;
  \end{itemize}

\end{frame}

\pgfdeclareimage{debug_buttons02}{img/debug/debug_buttons02.png}
\begin{frame}{Debugging (V)}

  \begin{center}
    \pgfuseimage{debug_buttons02}
  \end{center}

\end{frame}

\begin{frame}{Debugging (VI)}

  Dal pannello \textbf{variabili} è possibile:
  \begin{itemize}
    \item ispezionare il valore della variabile;
    \item (click destro) impostare un nuovo valore per la variabile;
  \end{itemize}

\end{frame}

\pgfdeclareimage{debug_perspective_exit}{img/debug/debug_perspective_exit.png}
\begin{frame}{Debugging (VII)}

  Per uscira dal debugging, ovvero cambiare prospettiva e ritornare alla finestra ``standard''
  di Eclipse, si possono usare i pulsanti posti in altro a destra:

  \begin{center}
    \pgfuseimage{debug_perspective_exit}
  \end{center}
\end{frame}


\section{Esercizi}
\subsection[Esercizi]{Esercizi}

\begin{frame}{Consigli vari}
  \begin{enumerate}
    \item Iniziate dai programmi più semplici;
    \item I commenti sono importanti (ma ci sono 2 scuole di pensiero);
    \item la leggibilità del codice è importante (\url{https://www.python.org/dev/peps/pep-0020/});
  \end{enumerate}

\end{frame}

\begin{frame}{Esercizi}
  \begin{enumerate}
   \item Dichiarare e inizializzare un intero e stampare a video se è pari o dispari;
   \item Definire un programma che dati tre numeri $a$, $b$ e $x$ stabilisca se
	 $x \in [a, b]$;
   \item Definire un programma che dati tre numeri $a$, $b$ e $c$ stabilisca quale
	 è il massimo;
   \item Definire un programma che dati tre numeri $a$, $b$ e $c$ li stampi in ordine
	crescente;
   \item Definire un programma che dati due numeri $a$, $b$ restituisca la divisione
	 (intera) $a/b$ ed il resto di tale divisione;
    \end{enumerate}

\end{frame}


% \appendix
% \section*{Backup}
% \begin{frame}
\begin{Huge}
Backup 
\end{Huge}
\end{frame}


\end{document}
