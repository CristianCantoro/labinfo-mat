\begin{frame}{Programmazione Orientata agli Oggetti (I)}
  
  \begin{itemize}
   \item La Programmazione Orientata agli Oggetti (o \emph{Object Oriented Programming}, OOP, in inglese)
         è un paradigma di Programmazione basato sui concetti di \textbf{oggetto} e \textbf{classe}.
   \item In questa lezione introdurremo i concetti di base legati agli oggetti e li utilizzeremo
         come una \textbf{struttura dati}, ovvero una entità che permette di gestire un insieme di
         dati (detti \emph{attributi} o \emph{membri}).
  \end{itemize}

\end{frame}

\begin{frame}{Programmazione Orientata agli Oggetti (II)}
  
  Finora \textbf{\Red{ad eccezione delle \texttt{String}}} abbiamo usato delle variabili di uno dei
  tipi \textbf{primitivi} (o \emph{atomici}):
  \begin{itemize}
    \item \texttt{\Word{int}}
    \item \texttt{\Word{double}}
    \item \texttt{\Word{float}}
    \item \dots
  \end{itemize}
  

  utilizzando un tipo primitivo sappiamo che i dati verrano rappresentati in memeria in un 
  un modo predefinito:
  \begin{itemize}
   \item \texttt{\Word{int}} $\rightarrow$ 4 byte (32 bit) 
   \begin{itemize}
    \item min: $-2^{31}$;
    \item max: $2^{31} - 1$;
   \end{itemize}

   \item \texttt{\Word{double}} $\rightarrow$ 8 byte (64 bit)
   \begin{itemize}
    \item $max \equiv |min|$: $1.7976931348623157 \cdot 10^{308}$;
    \item $\varepsilon$: $4.9 \cdot 10^{-324}$;
   \end{itemize}
  \end{itemize}

\end{frame}

\begin{frame}[fragile]\frametitle{Programmazione Orientata agli Oggetti (III)}
  
  Immaginiamo di volere scrivere un programma che memorizza i dati (nome e cognome,
  luogo di nascita, età) di una persona.
  \begin{JavaCodePlain}[commandchars=\\!|]
  
  String nome1 = \String!"Alice Rossi"|;
  String luogo1 = \String!"Milano"|;
  \Jint eta1 = 19;
  
  String nome2 = \String!"Roberto Verdi"|;
  String cognome2 = \String!"Roma"|;
  \Jint  eta2 = 20;
  \dots

  \end{JavaCodePlain}
  
  Problemi di questo approccio:
  \begin{itemize}
   \item   È molto scomodo creare nuove istanze dell'entità (concettuale) ``Persona'';
   \item   È molto facile commettere errori;
  \end{itemize}

\end{frame}

\begin{frame}[fragile]\frametitle{Programmazione Orientata agli Oggetti (IV)}

  Una \textbf{oggetto}:
  \begin{itemize}
    \item è un'\emph{entità software} costituito da un insieme di stati e comportamenti che si riferiscono a uno stesso concetto; \\
          {\footnotesize (\emph{«An object is a software bundle of related state and behavior.»}) }
          {\scriptsize \LightBlue{\url{https://docs.oracle.com/javase/tutorial/java/concepts/}} }
    \item consiste in una regione di memoria allocata;

  \end{itemize}
\end{frame}

\begin{frame}[fragile]\frametitle{Programmazione Orientata agli Oggetti (V)}

  Una \textbf{classe}:
  \begin{itemize}
    \item è un prototipo a partire dalla quale gli oggetti vengono \textbf{creati} (o \textbf{istanziati}, 
	  o \textbf{instanziati});\\
          {\footnotesize (\emph{«A class is a blueprint or prototype from which objects are created.»}) }
          {\scriptsize \LightBlue{\url{https://docs.oracle.com/javase/tutorial/java/concepts/}} }
    \item è la specifica che descrive quali sono i possibili stati e comportamenti di un oggetto creati
          a partire da esso;
    \item è un insieme di variabili (anche dette \textbf{\emph{membri}} o \textbf{\emph{attributi}}) e di
         funzioni i metodi (dette \textbf{\emph{metodi}});
    \item è anche chiamata \emph{Abstract Data Type} (ADT).
  \end{itemize}
\end{frame}


\pgfdeclareimage[width=0.35\paperwidth]{people01}{img/people_jumping_01.png}
\pgfdeclareimage[width=0.35\paperwidth]{people02}{img/people_jumping_02.png}
\pgfdeclareimage[width=0.35\paperwidth]{people03}{img/people_jumping_03.png}

\begin{frame}[fragile]\frametitle{Programmazione Orientata agli Oggetti (VI)}


  \begin{columns}[T]
    \begin{column}[T]{7cm}
    Classe:
    \begin{JavaCodePlain}[commandchars=\\!|]
    
    \Jpublic \Jclass Persona {

      \Jprivate String nome;      
      \Jprivate String luogoNascita;
      \Jprivate \Jint eta;

    }

    \end{JavaCodePlain}
    \end{column}

    \begin{column}[T]{5cm}
    Oggetti:
    \begin{center}
      \pgfuseimage{people01}
    \end{center}
    \end{column}
  \end{columns}


  \begin{minipage}[b]{12cm}
  {\scriptsize Photo credit: CC-BY-SA Mark Sebastian @ Wikimedia Commons \url{http://bit.ly/1kip83N}}
  \end{minipage}
\end{frame}

\begin{frame}[fragile]\frametitle{Programmazione Orientata agli Oggetti (VI)}


  \begin{columns}[T]
    \begin{column}[T]{7cm}
    Classe:
    \begin{JavaCodePlain}[commandchars=\\!|]
    
    \Jpublic \Jclass Persona {

      \Jprivate String nome;      
      \Jprivate String luogoNascita;
      \Jprivate \Jint eta;

    }
    
    \dots
    
    Persona Alice;

    \end{JavaCodePlain}
    \end{column}

    \begin{column}[T]{5cm}
    Oggetti:
    \begin{center}
      \pgfuseimage{people02}
    \end{center}
    \end{column}
  \end{columns}

\end{frame}

\begin{frame}[fragile]\frametitle{Programmazione Orientata agli Oggetti (VI)}


  \begin{columns}[T]
    \begin{column}[T]{7cm}
    Classe:
    \begin{JavaCodePlain}[commandchars=\\!|]
    
    \Jpublic \Jclass Persona {

      \Jprivate String nome;      
      \Jprivate String luogoNascita;
      \Jprivate \Jint eta;

    }

    \dots
    
    Persona Alice;
    Persona John;

    \end{JavaCodePlain}
    \end{column}

    \begin{column}[T]{5cm}
    Oggetti:
    \begin{center}
      \pgfuseimage{people03}
    \end{center}
    \end{column}
  \end{columns}
  
\end{frame}

\begin{frame}[fragile]\frametitle{Programmazione Orientata agli Oggetti (VI)}


  \begin{columns}[T]
    \begin{column}[T]{7cm}
    Classe:
    \begin{JavaCodePlain}[commandchars=\\!|]
    
    \Jpublic \Jclass Persona {

      \Jprivate String nome;      
      \Jprivate String luogoNascita;
      \Jprivate \Jint eta;

    }

    \dots
    
    Persona Alice;
    Persona John;

    \end{JavaCodePlain}
    \end{column}

    \begin{column}[T]{5cm}
    Oggetti:
    \begin{center}
      \pgfuseimage{people03}
    \end{center}
    \end{column}
  \end{columns}

  \begin{alertblock}{Oggetti vs. Classi}
    Un oggetto è un'istanza di una classe
  \end{alertblock}
  
\end{frame}

\begin{frame}[fragile]\frametitle{Classi vs. Oggetti (I)}

  \begin{columns}
    \begin{column}{6cm}
      Classe:
      \begin{itemize}
	\item descrizione delle proprietà \emph{comuni} a una tipologia di variabili;
	\item \`e un ``concetto'';
	\item \`e una parte di un programma;
      \end{itemize}
      
%       \begin{enumerate}
%        \item Persona
%        \item Album
%        \item Canzoni
%       \end{enumerate}
    \end{column}

    \begin{column}{6cm}
      Oggetti:
      \begin{itemize}
	\item rappresentazioni delle proprietà di una singola istanza;
	\item Un ``fenomeno''/``manifestazione''.
	\item una parte dei dati nell'esecuzione di un programma.
      \end{itemize}
      
%       \begin{enumerate}
%        \item Hillary Clinton, Rafael Nadal, Lewis Hamilton;
%        \item Thriller, Back in Black, The Dark Side of the Moon
%        \item {\scriptsize Thriller, Beat it, Billie Jean, Hells Bells, Shoot to Thrill, Back in Black, \dots}
%       \end{enumerate}
    \end{column}
  \end{columns}

  ${}$
  \begin{columns}[T]
    \begin{column}[T]{6cm}
      \begin{enumerate}
       \item Persona
       \item Album
       \item Canzoni
      \end{enumerate}
    \end{column}

    \begin{column}[T]{6cm}    
      \begin{enumerate}
       \item Hillary Clinton, Rafael Nadal, Lewis Hamilton;
       \item Thriller, Back in Black, The Dark Side of the Moon
       \item {\small Thriller, Beat it, Billie Jean, Hells Bells, Shoot to Thrill, Back in Black, \dots}
      \end{enumerate}
    \end{column}
  \end{columns}

\end{frame}

\pgfdeclareimage[width=0.25\paperwidth]{people01}{img/muffins.jpg}
\begin{frame}[fragile]\frametitle{Classi vs. Oggetti (II)}

  \begin{itemize}
   \item Nella programmazione ad oggetti (OOP) si scrivono classi;
    \begin{itemize}
      \item il codice sorgente che scriviamo contiene delle \textbf{classi};
      \item una classe è ``statica'';
      \item ``Una'';
    \end{itemize}
   \end{itemize}

   \begin{itemize}
   \item Gli oggetti sono creati a partire dalle classi;
    \begin{itemize}
      \item una classe è come una ricetta di una torta, gli oggetti sono le torte prodotte
      a partire dalla ricetta.
      \item un oggetto è ``dinamico''
      \item ``Molti''
    \end{itemize}
  \end{itemize}

  ${}$
  \begin{columns}[T]
    \begin{column}[T]{4cm}
      Ricetta (Classe) \textbf{Muffin}:
      \begin{enumerate}
       \item 3 uova;
       \item 380 g di farina;
       \item \dots
      \end{enumerate}
    \end{column}

    \begin{column}[T]{7cm}    
     \pgfuseimage{people01}
    \end{column}
  \end{columns}
  ${}$
  {\scriptsize Photo credit: CC-BY Sara K @ Flickr \url{http://bit.ly/1kioQdd}}
\end{frame}