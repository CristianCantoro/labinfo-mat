
\begin{frame}[fragile]\frametitle{Esercizi (I)}

  Creare un nuovo progetto chiamato \texttt{ProgettoMusica} che:
  \begin{itemize}
   \item contenga una classe principale \texttt{Raccolta} (\texttt{Raccolta.java});
   \item contenga una classe secondaria \texttt{Canzone} (\texttt{Canzone.java});
   \item la classe \texttt{Canzone} contiene i seguenti attributi \Jprivate{}:
   \begin{itemize}
    \item una variabile \texttt{nome} di tipo \texttt{String} (nome della canzone);
    \item una variabile \texttt{artista} di tipo \texttt{String} (nome dell'autore della canzone)
    \item una variabile \texttt{durata} di tipo \texttt{String} (durata della canzone)
    {\footnotesize (scrivere la durata come una stringa del tipo \texttt{XmYYs}, ovvero \texttt{X} minuti e \texttt{YY} secondi)};
    \item una variabile \texttt{rating} di tipo \texttt{\Jint} (voto della canzone da 1 a 5);
    \end{itemize}
  \end{itemize}
\end{frame}

\begin{frame}[fragile]\frametitle{Esercizi (II)}

  La classe \texttt{Canzone} ha:
  \begin{itemize}
   \item un costruttore con parametri che imposta i valori di \texttt{nome}, \texttt{artista}, \texttt{durata} e \texttt{rating};
   \item un costruttore senza parametri che imposta i valori di \texttt{nome} a \String{"Nessun nome"}, \texttt{nome} a \String{"Nessuno"}, 
   \texttt{durata} a \String{"0m00s"} e \texttt{rating} a 0;
   \item i metodi getter e setter per le variabili;
   \item un metodo \texttt{toString()} che stampa i dati contenuti nella classe come segue:
   \begin{itemize}
    \item \texttt{$<nome>$ di $<artista>$ ($<durata>$), $<rating>$ stelle}
    \item \texttt{Thriller di Michael Jackson (5m58s), 5 stelle}
   \end{itemize}
  \end{itemize}

\end{frame}

\begin{frame}{Esercizi (III)}

  \begin{itemize}
   \item Nella classe principale \texttt{Raccolta} creare un vettore di 5 canzoni (oggetti di tipo \texttt{Canzone})
   inizializzandoli con dati a vostra scelta e successivamente stamparli;
   \item Successivamente, modificare il programma per stampare solo le canzoni che hanno un rating superiore a 3;
   \item Infine, modificare nuovamente il programma per stampare solo le canzoni che hanno \texttt{\String{"Michael Jackson"}};
  \end{itemize}

\end{frame}

\begin{frame}[fragile]\frametitle{Esercizi (IV)}
  
  \textbf{Suggerimento:}
  \begin{itemize}
   \item Ricordate che per verificare l'uguaglianza tra due stringhe dovete usare il metodo \texttt{\textbf{equals}};
   \item Ricordate che c'è differenza tra maiuscole minuscole;
  \end{itemize}
  \begin{JavaCodePlain}[commandchars=\\!|]
  
  String mj = \String!"Michael Jackson"|;
  
  \Jcomment!mj.equals(c1.getArtista()) ritornerà True|
  Canzone c1 = \Jnew Canzone(\String!"Thriller"|, \String!"Michael Jackson"|,
                           \String!"5m58"|, 5);
  
  \Jcomment!mj.equals(c2.getArtista()) ritornerà False|
  Canzone c2 = \Jnew Canzone(\String!"Back in Black"|, \String!"AC/DC"|,
                           \String!"4m15"|, 5);
  \end{JavaCodePlain}

\end{frame}

