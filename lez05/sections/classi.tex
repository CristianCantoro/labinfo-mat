
\begin{frame}[fragile]\frametitle{Classi: costruttore (I)}

  Una classe necessita di un \textbf{costruttore} per potere essere istanziata:
  
  \begin{JavaCodePlain}[commandchars=\\!|]
  \Jpublic \Jclass Persona {
    \Jcomment!attributi della classe -> variabili dell'oggetto|
    \Jprivate String nome;      
    \Jprivate String luogoNascita;
    \Jprivate \Jint eta;

    \Jcomment!Costruttore con parametri|
    \Jpublic Persona(String nome, String luogoNascita, \Jint eta) {
      \dots
    }

    \Jcomment!Costruttore senza parametri con valori di default|
    \Jpublic Persona() {
      \dots
    }    
  }
  \end{JavaCodePlain}

\end{frame}

\begin{frame}[fragile]\frametitle{Classi: costruttore (II)}

  Il \textbf{costruttore} è il \emph{metodo} che crea la classe:
  
  \begin{JavaCodePlain}[commandchars=\\!|]
  \Jpublic \Jclass Persona {
    \dots
    \Jcomment!Costruttore con parametri|
    \Jpublic Persona(String nome, String luogoNascita, \Jint eta) {
      \Jthis.nome = nome;
      \Jthis.luogoNascita = luogo;
      \Jthis.eta = eta;
    }

    \Jcomment!Costruttore senza parametri con valori di default|
    \Jpublic Persona() {
      \Jthis.nome = \String!"Anonimo"|;
      \Jthis.luogoNascita = \String!"Sconosciuto"|;
      \Jthis.eta = -1;
    }
  }
  \end{JavaCodePlain}

\end{frame}

\begin{frame}[fragile]\frametitle{Classi: costruttore (III)}

  La parola riservata \Jthis{} serve per specificare se ci si sta riferendo
  ai parametri del costruttore o ai membri della classe.
  
  \begin{JavaCodePlain}[commandchars=\\!|]
  \dots
    \Jpublic Persona(String nome, String luogoNascita, \Jint eta) {
      \Jthis.nome = nome;
      \Jthis.luogoNascita = luogo;
      \Jthis.eta = eta;
  \dots
  \end{JavaCodePlain}

\end{frame}

\begin{frame}[fragile]\frametitle{La parola riservata \texttt{this}}
  
  \begin{itemize}
   \item Costruttore con parametri:
  \end{itemize}
  \begin{JavaCodePlain}[commandchars=\\!|]
    \Jpublic Persona(String nome, String luogoNascita, \Jint eta) {
      \Jthis.nome = nome;
      \Jthis.luogoNascita = luogo;
      \Jthis.eta = eta;
  
  Persona anna = Persona(\String!"Anna Rossi"|, \String!"Milano"|, 19);
  \end{JavaCodePlain}

  \begin{itemize}
   \item Costruttore senza parametri:
  \end{itemize}
  \begin{JavaCodePlain}[commandchars=\\!|]
    \Jpublic Persona() {
      \Jthis.nome = \String!"Anonimo"|;
      \Jthis.luogoNascita = \String!"Sconosciuto"|;
      \Jthis.eta = -1;
    }
  
  Persona anon = Persona();
  \end{JavaCodePlain}

\end{frame}

\begin{frame}{Accessibilità delle variabili (I)}

  Le parole riservate \Jprivate, \Jpublic, \Jprotected{} modificano l'accessibilità delle variabili:
  \begin{enumerate}
   \item \Jpublic: sono accessibili da chiunque. (\Green{\textbf{più accessibili}})
   \item \Jprotected: sono accessibili solo all'interno dallo stesso package dalla stessa classe e dalle sottoclassi.
   \item default (nessun modificatore aggiuntivo): sono accessibili solo all'interno dallo stesso package dalla stessa classe.
   \item \Jprivate: sono accessibili sollo dalla stessa classe. (\Red{\textbf{meno accessibili}})
  \end{enumerate} 

  \pause{
  \begin{alertblock}{public e private}
    Noi useremo solo \textbf{public} (\Green{\textbf{più accessibili}}) e \textbf{private} (\Red{\textbf{meno accessibili}})
  \end{alertblock}
  }

\end{frame}

\begin{frame}{Accessibilità delle variabili (II)}

%   \begin{table}[]
%   \centering
%     \begin{tabular}{l|llll}
%       Modificatore                                                        & \multicolumn{1}{c}{Classe} & \multicolumn{1}{c}{Package} & \multicolumn{1}{c}{Sottoclasse} & \multicolumn{1}{c}{\begin{tabular}[c]{@{}c@{}}Resto del\\ mondo\end{tabular}} \\ \hline
%       \begin{tabular}[c]{@{}l@{}}\textbf{public}\\${}$\end{tabular}           & Sì                         & Sì                          & Sì                              & Sì                                                                            \\
%       \begin{tabular}[c]{@{}l@{}}protected\\${}$\end{tabular}                 & Sì                         & Sì                          & Sì                              & No                                                                            \\
%       \begin{tabular}[c]{@{}l@{}}default\\ (no modificatore)\end{tabular} & Sì                         & Sì                          & No                              & No                                                                            \\
%       \begin{tabular}[c]{@{}l@{}}\textbf{private}\\${}$\end{tabular}          & Sì                         & No                          & No                              & No                                                                            
%     \end{tabular}
%   \end{table}  

  La tabella seguente riassume l'accessibilità delle variabili dichiarate con i vari modificatori.
  \begin{table}[htbp]
    \begin{tabular}{|l|cccc|}
      \hline
      \multicolumn{1}{|c|}{\textbf{Modificatore}} & \textbf{Class} & \textbf{Package} & \textbf{Sottoclassi} & \textbf{Mondo} \\ \hline
      \rowcolor[HTML]{9AFF99}
      \textbf{public} & \textbf{Y} & \textbf{Y} & \textbf{Y} & \textbf{Y} \\ \hline
      protected & Y & Y & Y & N \\ \hline
      default
      (no modificatore) & Y & Y & N & N \\ \hline
      \rowcolor[HTML]{FD6864}
      \textbf{private} & \textbf{Y} & \textbf{N} & \textbf{N} & \textbf{N} \\ \hline
    \end{tabular}
  \end{table}
  {\footnotesize Si veda anche:}\\
  {\scriptsize \LightBlue{\url{https://docs.oracle.com/javase/tutorial/java/javaOO/accesscontrol.html}} }

\end{frame}

\begin{frame}{Accessibilità delle variabili: metodi getter e setter (I)}

  Dato che le variabili \Jprivate{} non sono accessibili al di fuori della classe è necessario creare dei metodi
  che restituiscono e modificano il loro valore. Questi metodi sono detti rispettivamente:
  \begin{itemize}
   \item \textbf{get}ter: restitituisce (\textbf{get}) il valore della variabile, ad es. \texttt{getNome()};
   \item \textbf{set}ter: imposta (\textbf{set}) il valore della variabile, ad es. \texttt{setNome()};
  \end{itemize}
  di solito i nome sono \texttt{getNomeVariabile()} e \texttt{setNomeVariabile()};
   
  \pause{
  \begin{alertblock}{Variabili private vs. public}
    Se una variabile è \Jpublic{} non è necessario avere getter e setter, ma è buona norma usare 
    variabili \Jprivate{} nelle classi
  \end{alertblock}
  }

\end{frame}

\begin{frame}[fragile]\frametitle{Accessibilità delle variabili: metodi getter e setter (II)}

  Esempio di metodi \textbf{get}ter e \textbf{set}ter per la variabile \texttt{nome} nella classe
  Persona:
  
  \begin{JavaCodePlain}[commandchars=\\!|]
  \Jpublic \Jclass Persona {

    \Jprivate String nome;
    \dots
    
    \Jpublic String getNome() {
      \Jreturn nome;
    }

    \Jpublic \Jvoid setNome(String nome) {
      \Jthis.nome = nome;
    }

  }
  \end{JavaCodePlain}

\end{frame}

\begin{frame}{Metodo \texttt{toString} (I)}

  \begin{itemize}
    \item il metodo \texttt{toString()} restituisce una \textbf{stringa} ovvero la ``rappresentazione testuale''
	  dell'oggetto su cui è invocato (da usare ad esempio quando si stampa l'oggetto). 
    \item la definizione default del metodo nella classe Object è \Blue{\texttt{$<nome\_classe>@<hashcode>$}};
    \item possiamo ridefinire questo metodo per stampare un oggetto nel modo che vogliamo;
  \end{itemize}

\end{frame}

\begin{frame}[fragile]\frametitle{Metodo \texttt{toString} (II)}

  Esempio:
  \begin{JavaCodePlain}[commandchars=\\!|]
  \Jpublic \Jclass Persona {
  
    \Jprivate String nome;
    \Jprivate String luogoNascita;
    \Jprivate \Jint eta;

    \dots

    \Jpublic String toString() {
      \Jreturn nome + \String!" ("| + luogoNascita + \String!"), "| + eta;
    }
  }
  \end{JavaCodePlain}

  \begin{itemize}
   \item[$\Rightarrow$] \JprintLN(anna); $\Longrightarrow$ \texttt{Anna Rossi (Milano), 19}
  \end{itemize}

\end{frame}

\begin{frame}{Accessibilità delle classi}

  A livello delle classe \`e possibile usare:
  \begin{enumerate}
   \item \Jpublic: classe accessibile anche da altri package.
   \item default (nessun modificatore aggiuntivo): classe accessibile solo all'interno dello stesso package.
  \end{enumerate}

\end{frame}

\begin{frame}{Struttura di un progetto (I)}

  In un progetto esistono più classi:
  \begin{itemize}
   \item Una classe principale che contiene il metodo \textbf{main};
   \item altri file contenenti le altre classi ``secondarie'': create un nuovo file per ogni classe
   (ad esempio \texttt{Persona.java} per la classe \texttt{Persona});
   \item il nome di una classe ha l'iniziale maiuscola e si usa la notazione CamelCase/PascalCase;
   \item non si possono avere due classi con lo stesso nome all'interno dello stesso progetto;
  \end{itemize}
  
\end{frame}

\begin{frame}[fragile]\frametitle{Struttura di un progetto (II)}

  Esempio:
  \begin{itemize}
   \item creiamo un progetto EsempioProgetto che riguarderà la musica (album e canzoni);
   \item La classe principale si chiama \texttt{EsempioProgetto} e si trova nel file \texttt{EsempioProgetto.java};
   \item Esistono due classi \texttt{Album} (\texttt{Album.java}) e \texttt{Canzone} (\texttt{Canzone.java}) che possono
         essere usate dentro \texttt{EsempioProgetto}.
  \end{itemize}

\end{frame}

\pgfdeclareimage[width=0.35\paperwidth]{progetto}{img/progetto.png}
\begin{frame}[fragile]\frametitle{Struttura di un progetto (II)}

  \begin{center}
    \pgfuseimage{progetto}
  \end{center}

  Classe:
  \begin{JavaCodePlain}[commandchars=\\!|]
  
  \Jpublic \Jclass EsempioProgetto {
  
    \Jpublic \Jstatic \Jvoid main(String[] \Jargs) {
    
      Album thriller = \Jnew Album(\String!"Thriller"|, \dots);
      Canzone hb = new Canzone(\String!"Hells Bells"|, \String!"AC/DC"|, \String!"5:13"|);
    }
  }
  \end{JavaCodePlain}

\end{frame}
