\DeclareMathOperator*{\len}{\textbf{length of }}

\newcommand{\btVFill}{\vskip0pt plus 1filll}

\setbeamercolor{structure}{fg=cyan!90!black}

\newenvironment{variableblock}[3]{%
\setbeamercolor{block body}{#2}
\setbeamercolor{block title}{#3}
\begin{block}{#1}}{\end{block}}

\lstdefinestyle{JavaPlain}{ %
basicstyle=\scriptsize\ttfamily, % the size of the fonts 
numbers=left,                   % where to put the line-numbers
numberstyle=\tiny,      % the size of the fonts that are used for th
stepnumber=1,                   % the step between two line-numbers
numbersep=5pt,                  % how far the line-numbers are from the code
backgroundcolor=\color{white},  % choose the background color
showspaces=false,               % show spaces adding particular underscores
showstringspaces=false,         % underline spaces within strings
showtabs=false,                 % show tabs within strings adding 
frame=single,           % adds a frame around the code
tabsize=2,          % sets default tabsize to 2 spaces
captionpos=b,           % sets the caption-position to bottom
breaklines=true,        % sets automatic line breaking
breakatwhitespace=false,    % sets if automatic breaks should only happen
fancyvrb=true,
fvcmdparams=textbf 1 textit 1,
}

% Colors
\definecolor{ballblue}{rgb}{0.13, 0.67, 0.8}
\definecolor{brown(web)}{rgb}{0.65, 0.16, 0.16}
\definecolor{brown(traditional)}{rgb}{0.59, 0.29, 0.0}

\newcommand\Red[1]{\textcolor{red}{#1}}
\newcommand\Green[1]{\textcolor{green!50!black}{#1}}
\newcommand\LightGreen[1]{\textcolor{green!60!black}{#1}}
\newcommand\Blue[1]{\textcolor{blue!60!white}{#1}}
\newcommand\Violet[1]{\textcolor{violet}{#1}}
\newcommand\LightBlue[1]{\textcolor{ballblue}{#1}}
\newcommand\Grey[1]{\textcolor{gray}{#1}}
\newcommand\Gray[1]{\textcolor{gray}{#1}}
\newcommand\Brown[1]{\textcolor{brown(web)}{#1}}
\newcommand\LightBrown[1]{\textcolor{brown(traditional)}{#1}}

% Helpers
\newcommand\Jcomment[1]{\LightGreen{// #1}}
\newcommand\JcommentMulti[1]{\LightGreen{/* #1}}
\newcommand\String[1]{\textcolor{blue!80!blue}{#1}}
\newcommand\Word[1]{\textcolor{purple!90!red}{#1}}
\newcommand\bang{!}
\newcommand\pipe{\|}

% Prints
% \newcommand\Jprintf[2][]{\LightBlue{printf}(\String{#1}, #2)}
\newcommandtwoopt{\Jprintf}[2][-NoValue-][-NoValue-]{%
    \ifthenelse{\equal{#2}{-NoValue-}}{\LightBlue{printf}(\String{#1})}{\LightBlue{printf}(\String{#1}, #2)}%
}

\newcommand\Jprintln[1]{\LightBlue{println}(\String{#1})}
\newcommand\JprintLN{\LightBlue{println}}

% Control structures
% Shortcut for JavaFor:
% \JavaFor (\Blue!init| \Green!n <= Math.sqrt(numero);| \Violet!n++|)
\newcommand\JavaForColors[3]{\Blue{#1}; \Green{#2}; \Violet{#3}}

\newcommand\Jfor[1][-NoValue-]{%
  \ifthenelse{\equal{#1}{-NoValue-}}{\Word{for}}{\JavaForO[#1]}%
}

\newcommandtwoopt{\JavaForO}[3][-NoValue-][-NoValue-]{%
    \JavaForOptions{#1}{#2}{#3}%
}

\DeclareDocumentCommand\JavaForOptions{ ggg }{%
  \Word{for}\IfValueT{#1}{%
    (\JavaForColors{#1}{#2}{#3}) %
  }%
}

% Shortcut for JavaFor:
% \JavaFor (\Blue!init| \Green!n <= Math.sqrt(numero);| \Violet!n++|)
\newcommand\Jif[1][-NoValue-]{%
  \ifthenelse{\equal{#1}{-NoValue-}}{\Word{if}}{\Word{if} (\Green{#1})}%
}

% Other reserved words
\newcommand\Jstatic{\Word{static}}
\newcommand\Jpublic{\Word{public}}
\newcommand\Jprivate{\Word{private}}
\newcommand\Jdouble{\Word{double}}
\newcommand\Jfloat{\Word{float}}
\newcommand\Jint{\Word{int}}
\newcommand\Jvoid{\Word{void}}
\newcommand\Jreturn{\Word{return}}
\newcommand\Jclass{\Word{class}}
\newcommand\Jargs{\LightBrown{args}}

\newenvironment{JavaCodePlain}[1][]
  { \VerbatimEnvironment%
    \begin{Verbatim}[#1]}
  { \end{Verbatim}  } 

\algdef{SE}[DOWHILE]{Do}{DoWhile}{\algorithmicdo}[1]{\algorithmicwhile\ #1}%
\renewcommand{\algorithmicrequire}{\textbf{Input:}}
\renewcommand{\algorithmicensure}{\textbf{Output:}}