
\begin{frame}{Metodo della bisezione}

  Negli esercizi della scorsa volta abbiamo calcolato gli zeri di una \textbf{funzione}:
  \begin{equation}
    f(x) = x^3 - x - 2
  \end{equation}

\end{frame}


\begin{frame}[fragile]\frametitle{Funzioni (I)}

  Possiamo definire delle funzioni per rendere il codice:
  \begin{itemize}
    \item più \textbf{leggibile}
    \item più \textbf{riutilizzabile}
    \item più \textbf{compatto} e semplice da modificare
  \end{itemize}
\end{frame}

\begin{frame}[fragile]\frametitle{Funzioni (II)}

  Dichiarazione e implementazione di una funzione:

  \begin{JavaCodePlain}[commandchars=\\!|]
  \Jpublic \Jstatic \Jdouble f(\Jdouble x) {
      \Jreturn(x * x * x - x - 2);
  }
  \end{JavaCodePlain}     
  analizziamo la sintassi parola per parola.

\end{frame}

\begin{frame}[fragile]\frametitle{Funzioni (III)}

  Dichiarazione e implementazione di una funzione:
  
  \begin{JavaCodePlain}[commandchars=\\!|]
  \Red!public| \Red!static| \Grey!double f(double x) {|
      \Grey!return(x * x * x - x - 2);|
  \Grey!}|
  \end{JavaCodePlain}
  
  Le parole chiave \Red{public} e \Red{static} modificano la
  visibilità della funzione. Per ora ignoratele e scrivete
  sempre così.

\end{frame}


\begin{frame}[fragile]\frametitle{Funzioni (IV)}

  Dichiarazione e implementazione di una funzione:
  
  \begin{JavaCodePlain}[commandchars=\\!|]
  \Grey!public static| \Red!double| \Grey!f(double x) {|
      \Grey!return(x * x * x - x - 2);|
  \Grey!}|
  \end{JavaCodePlain}
  
  Questo è il \textbf{tipo del valore di ritorno} (\emph{return type})
  della funzione, vuol dire, in questo caso, che la funzione restituisce
  un double.

\end{frame}

\begin{frame}[fragile]\frametitle{Funzioni (V)}

  Dichiarazione e implementazione di una funzione:

  \begin{JavaCodePlain}[commandchars=\\!|]
  \Grey!public static| \Grey!double| \Red!f|\Grey!(double x) {|
      \Grey!return(x * x * x - x - 2);|
  \Grey!}|
  \end{JavaCodePlain}
  
  Dopo il tipo di ritorno si specifica il \textbf{nome} della funzione,
  come per le variabili si tratta dell'identificativo da utilizzare per
  utilizzare una funziona.

\end{frame}

\begin{frame}[fragile]\frametitle{Funzioni (VI)}

  Dichiarazione e implementazione di una funzione:

  \begin{JavaCodePlain}[commandchars=\\!|]
  \Grey!public static| \Grey!double| \Grey!f|\Red!(|\Grey!double x|\Red!)| \Grey!{|
      \Grey!return(x * x * x - x - 2);|
  \Grey!}|
  \end{JavaCodePlain}

  Le funzioni sono caratterizzate dalle parentesi tonde \texttt{f()} che servono
  per potere \emph{chiamare} una funzioone.

\end{frame}

\begin{frame}[fragile]\frametitle{Funzioni (VII)}

  Dichiarazione e implementazione di una funzione:

  \begin{JavaCodePlain}[commandchars=\\!|]
  \Grey!public static| \Grey!double| \Grey!f(|\Red!double x|) \Grey!{|
      \Grey!return(x * x * x - x - 2);|
  \Grey!}|
  \end{JavaCodePlain}

  All'interno delle parentesi tonde si indicano gli argometi della
  funzione detti \textbf{parametri formali}.

\end{frame}

\begin{frame}[fragile]\frametitle{Funzioni (VIII)}

  Dichiarazione e implementazione di una funzione:

  \begin{JavaCodePlain}[commandchars=\\!|]
  \Grey!public static| \Grey!double| \Grey!f(|\Red!double| \Grey!x) {|
      \Grey!return(x * x * x - x - 2);|
  \Grey!}|
  \end{JavaCodePlain}

  Ogni \emph{parametro formale} è caratterizzato da un proprio  \textbf{tipo}...

\end{frame}

\begin{frame}[fragile]\frametitle{Funzioni (IX)}

  Dichiarazione e implementazione di una funzione:

  \begin{JavaCodePlain}[commandchars=\\!|]
  \Grey!public static| \Grey!double| \Grey!f(double| \Red!x|\Grey!) {|
      \Grey!return(x * x * x - x - 2);|
  \Grey!}|
  \end{JavaCodePlain}

  e da un proprio \textbf{nome}.

\end{frame}

\begin{frame}[fragile]\frametitle{Funzioni (X)}

  Dichiarazione e implementazione di una funzione:

  \begin{JavaCodePlain}[commandchars=\\!|]
  \Grey!public static| \Grey!double| \Grey!f(|\Red!double x1, int x2|\Grey!) {|
      \Grey!return(x1 * x1 * x1 - x2 - 2);|
  \Grey!}|
  \end{JavaCodePlain}

  Una funzione può avere più parametri formali ciascuno caratterizzato dal
  proprio tipo e nome.

\end{frame}

\begin{frame}[fragile]\frametitle{Funzioni (XI)}

  Dichiarazione e implementazione di una funzione:
  
  \begin{JavaCodePlain}[commandchars=\\!|]
  \Grey!public static double f(double x) |\Red!{|
      \Red!return(x * x * x - x - 2);|
  \Red!}|
  \end{JavaCodePlain}

  Le parentesi graffe \texttt{\{\}} delimitano il \textbf{corpo}
  (o \emph{scope}) della funzione.

\end{frame}

\begin{frame}[fragile]\frametitle{Funzioni (XII)}

  Dichiarazione e implementazione di una funzione:
  \begin{JavaCodePlain}[commandchars=\\!|]
  \Grey!public static double f(double x) {|
      \Red!return|\Grey!(x * x * x - x - 2);|
  \Grey!}|
  \end{JavaCodePlain}

  La parola riservata \Red{return} indica quale valore sarà restituito dalla funzione.

\end{frame}

\begin{frame}[fragile]\frametitle{Funzioni: riassumendo}

  Dichiarazione e implementazione di una funzione:
  \begin{JavaCodePlain}[commandchars=\\!|]
  \Red!public static| \Green!double| \Violet!f|(\Blue!double x|) {
      \Brown!return|(x * x * x - x - 2);
  }
  \end{JavaCodePlain}

  \begin{itemize}
   \item \Red{\texttt{public static} $\rightarrow$ visibilità}
   \item \Green{\texttt{double} $\rightarrow$ tipo del valore di ritorno}
   \item \Violet{\texttt{f} $\rightarrow$ nome della funzione}
   \item \Blue{\texttt{double x} $\rightarrow$ parametri formali}
   \item \Brown{\texttt{return} $\rightarrow$ valore di ritorno}
  \end{itemize}

\end{frame}

\begin{frame}[fragile]\frametitle{Procedure}
  Le procedure sono funzioni che non restituiscono alcun valore.
  Per indicare questo fatto si usa come valore di ritorno \Word{\textbf{void}}:
  
  \begin{JavaCodePlain}[commandchars=\\!|]
  \Red!public static| \Green!\textbf!void|| \Violet!f|(\Blue!double x|) {
    \Jprintln!"Questa è una procedura"|;
  }
  \end{JavaCodePlain}
  {\scriptsize (Nelle slide userò \LightBlue{printf} e \LightBlue{println} invece di
   \LightBlue{System.out.printf} o \LightBlue{System.out.println}).}
 
  \begin{itemize}
   \item \Red{\texttt{public static} $\rightarrow$ visibilità}
   \item \Green{\textbf{\texttt{void}} $\rightarrow$ nessun valore ritornarto, è una \textbf{procedura}}
   \item \Violet{\texttt{f} $\rightarrow$ nome della procedura}
   \item \Blue{\texttt{double x} $\rightarrow$ parametri formali}
   \item \Brown{\texttt{return} $\rightarrow$ valore di ritorno}
  \end{itemize}
\end{frame}

\begin{frame}[fragile]\frametitle{Funzioni e procedure: esempio (I)}

  \begin{JavaCodePlain}[commandchars=\\!|]
  \Word!public class| FunzioniProcedure {

    \Jcomment!Questa è una funzione di due parametri x e y|
    \Jcomment!che restituisce un intero|
    \Word!public static int| somma(\Word!int| x, \Word!int| y) {
        \Word!return| x + y;
    }

    \dots
  \end{JavaCodePlain}


\end{frame}

\begin{frame}[fragile]\frametitle{Funzioni e procedure: esempio (II)}

  \begin{JavaCodePlain}[commandchars=\\!|]
    \dots

    \Jcomment!Questa è una procedura di due parametri x e y|
    \Word!public static void| stampaSomma(\Word!int| x, \Word!int| y) {

      \Jprintf["== Somma di due numeri ==\%n"];
      \Jprintf["Il primo parametro (x) vale \%d.\%n"][x];
      \Jprintf["Il secondo parametro (y) vale \%d.\%n"][y];
      \Jprintf["La somma di \%d e \%d è \%d.\%n"][x, y, x + y];

      \Jcomment!La riga seguente può essere omessa,|
      \Jcomment!io preferisco indicarla per chiarezza|
      \Word!return|;
    }
    
    \dots
  \end{JavaCodePlain}

\end{frame}

\begin{frame}[fragile]\frametitle{Funzioni e procedure: esempio (III)}

  \begin{JavaCodePlain}[commandchars=\\!|]
    \dots

    \Jcomment!Questo è il main, come al solito|
    \Word!public static void| main(String[] \LightBrown!args|) {
    
      \Jint a = 7;
      \Jint b = 11;
      \Jint risultato;
      
      \Jcomment!Chiamata della funzione|
      risultato = somma(a, b);
      \Jprintf["risultato: \%d.\%n"][risultato];
      
      \Jcomment!Chiamata della procedura|
      stampaSomma(a, b);
    }
  }
  \end{JavaCodePlain}

\end{frame}

\begin{frame}[fragile]\frametitle{Parametri attuali}

  Nella slide precedente abbiamo visto
  \begin{JavaCodePlain}[commandchars=\\!|]
  \Word!public static int| somma(\Word!int| x, \Word!int| y) {
  \dots
  \Word!public static void| stampaSomma(\Word!int| x, \Word!int| y) {
  \end{JavaCodePlain}
  \textbf{definizioni} della funzione \texttt{somma} e della procedura
  \texttt{stampaSomma}. Entrambe le definizioni hanno come \textbf{parametri formali} 
  le variabli \texttt{x} e \texttt{y}.\newline
  
  Le chiamate:
  \begin{JavaCodePlain}[commandchars=\\!|]
  risultato = somma(x, y);
  \dots
  stampaSomma(x, y);
  \end{JavaCodePlain}
  hanno come \textbf{parametri attuali} le variabli \texttt{a} e \texttt{b}.

\end{frame}

\begin{frame}[fragile]\frametitle{Parametri passati per valore (I)}

  \begin{JavaCodePlain}[commandchars=\\!|]
  \Jpublic \Jclass StampaParola {

    \Jcomment!di solito in Java si usa il CamelCase|
    \Jpublic \Jstatic \Jvoid stampaParolaInVerticale(String parola) {

      \Jcomment!String.toUpperCase() rende una stringa tutta|
      \Jcomment!maiuscola|
      parola = parola.toUpperCase();
      
      \Jcomment!String.length restituisce la lunghezza della parola|
      for(int i = 0; i < parola.length(); i++) {

	      \Jcomment!parola.charAt(i) restituisce l'i-esimo carattere|
	      \Jcomment!della parola si parte a contare da zero.|
	      \JprintLN(parola.charAt(i));
      }
    }
  \dots
  \end{JavaCodePlain}

\end{frame}

\begin{frame}[fragile]\frametitle{Parametri passati per valore (II)}

  \begin{JavaCodePlain}[commandchars=\\!|]
   \dots

    \Jpublic \Jstatic \Jvoid main(String[] args) {
    
      String parola = \String!"Ciao"|;
      
      \Jprintf["La parola da stampare è: \%s.\%n"][parola];
      stampaParolaInVerticale(parola);
      \Jprintf["La parola stampata è: \%s.\%n"][parola];

    }
  }
  \end{JavaCodePlain}

\end{frame}

\begin{frame}[fragile]\frametitle{Funzioni: diagramma}

%   \begin{tikzpicture}[node distance=.3\textwidth]
%     \node (x)  {$x$};
%     \node (y) [right of=x] {$y=f(x)$};
%     \draw[->,thick] (x) -- node[above] {$f$} (y);
%   \end{tikzpicture}

  Possiamo immaginare una funzione come una ``scatola nera'' che
  rende in ingresso i parametri e restituisce un valore:

  \begin{center}
    \begin{tikzpicture}[baseline=-0.5ex, on grid, node distance=2cm]
      \node (init) {$x_1, x_2, x_3, \dots$};
      \node[draw,right of=init, minimum size=1cm] (func) {$f$}; 
      \node[right= 3cm of func] (end) {$f(x_1, x_2, x_3, \dots)$};
      \node[below= 1cm of func] (def1) {\texttt{return\_type f}};
      \node[below= 0.5cm of def1] (def2) {\texttt{type1 $x_1$, type2 $x_2$, type3 $x_3$, $\dots$}};
      \draw[-] (init)--(func);
      \draw[-latex] (func)--(end);
    \end{tikzpicture}
  \end{center}
  
  \pause{
  \begin{alertblock}{Attenzione}
    In un linguaggio imperativo è possibile che una funzioni modifichi il valore di alcune variabili in ingresso!
  \end{alertblock}
  }

\end{frame}
