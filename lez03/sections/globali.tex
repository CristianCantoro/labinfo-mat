
\begin{frame}[fragile]\frametitle{Variabili globali e locali}

  All'interno di una funzione possono essere definite nuove \emph{variabili}.
  Queste variabili sono \textbf{locali} ovvero esistono e sono utilizzatibili 
  solo all'interno della funzione stessa.
  
  \begin{JavaCodePlain}[commandchars=\\!|]
  \dots
  \Word!public static int| somma(\Word!int| x, \Word!int| y) {
      \Jint ris;
      ris = x + y;
      \Word!return| ris;
  }

  \Word!public static void| main(String[] \LightBrown!args|) {    
    \Jint a = 7;
    \Jint b = 11;
    \Jint risultato;
    
    \Jcomment!Qui *non posso* usare ris\bang|
    \dots 
  \end{JavaCodePlain}

\end{frame}

\begin{frame}[fragile]\frametitle{Variabili globali e locali}

  Le parentesi graffe \texttt{\{\}} definiscono uno \textbf{scope}
  (lett. \emph{ambito}) ovvero un blocco di codice.

  \begin{JavaCodePlain}[commandchars=\\!|]

  \Word!public static void| main(String[] \LightBrown!args|) {    
    \Jint fatt  = 1;
    \Jint N = 10;

    \Jfor (\Jint n = 2; n < N; n++) {
      \Jcomment!La variabile n esiste solo all'interno di questo|
      \Jcomment!blocco di codice|
      \dots
    }
  
  \end{JavaCodePlain}

\end{frame}

\begin{frame}[fragile]\frametitle{Variabili globali e locali}

  Come avete già visto potete usare variabili che avete definito
  in blocco più esterno dentro blocchi più interni.

  \begin{JavaCodePlain}[commandchars=\\!|]

  \Word!public static void| main(String[] \LightBrown!args|) {    
    \Jint fatt  = 1;
    \Jint N = 10;

    \Jfor (\Jint n = 2; n < N; n++) {
      \Jcomment!fatt è definita nel blocco più esterno, quindi|
      \Jcomment!la posso usare anche qui|
      fatt = fatt * n;
    }
  
  \end{JavaCodePlain}

\end{frame}

\begin{frame}[fragile]\frametitle{Variabili globali e locali}

  Posso definere delle variabili che possano essere usate in ogni
  scope, queste si dicono variabili \textbf{globali}.

  \begin{JavaCodePlain}[commandchars=\\!|]
  \Word!public class| GlobaliLocali {
    
    \Jint numero = 0;
    
  \dots
  \end{JavaCodePlain}

  queste variabili possono essere modificate da ogni funzione e quindi
  \alert{vanno usate con attenzione}. È meglio limitare il più possibile
  lo scope di una variabile.
    
  Cercate di dare nomi significativi alle variabili globali altrimenti
  sembra che ``piovano dal cielo''. Solitamente le variabili globali
  (specie se rappresentano una costante) si scrivono con nomi
  tutti in \texttt{MAIUSCOLO}.
  
\end{frame}