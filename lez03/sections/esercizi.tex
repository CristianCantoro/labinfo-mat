\subsection[Esercizi]{Esercizi}

\begin{frame}[fragile]\frametitle{Esercizi (I)}

  \begin{itemize}
    \item inizializzare un array  di 10 interi e inserire in ciascun elemento
	  il valore 5 (utilizzare un ciclo for)
    \item inizializzare un  array  di  interi e assegnare a ciascun elemento
	  un valore uguale alla sua posizione 
    \item Dopo aver inizializzato l'array precedente, copiare il suo  contenuto 
	  in ordine nverso in un altro array della stessa dimensione.
    \begin{itemize}
      \item $(1, 2, 3) \rightarrow (3, 2, 1)$
    \end{itemize}
  \end{itemize}

  Suggerimento, con \texttt{arr.\Blue{length}} si può ottenere la lunghezza dell'array:
  \begin{JavaCodePlain}[commandchars=\\!|]
    \Jint arr[]
    arr = \Word!new| \Jint[10];
    
    \Jfor (\Jint i = 0; i < arr.length; i++) {
      \dots
    }  
  \end{JavaCodePlain}

\end{frame}

\begin{frame}{Esercizi (I)}

  \begin{itemize}
  \item Implementare una funzione che dato in ingresso un vettore di numeri interi
	restituisca quel vettore ordinato in ordine crescente.
  \end{itemize}

  Esistono tantissimi algoritmi di ordinamento, ecco l'idea di base per uno di essi:
  \begin{itemize}
   \item array $v$ di lunghezza $N$, si fa scorrere l'indice $i$ da  $0$ a $N-1$ ripetendo i seguenti passi:
   \begin{enumerate}
    \item si cerca il più piccolo elemento della sottosequenza $v[i..N]$
    \item si scambia questo elemento con l'elemento $i$-esimo
   \end{enumerate}
  \end{itemize}

  Un'animazione descrittiva è disponibile a questo indirizzo:
  {\scriptsize \url{https://upload.wikimedia.org/wikipedia/commons/9/94/Selection-Sort-Animation.gif}}

\end{frame}

\begin{frame}{Esercizi (II)}

  \begin{itemize}
   \item Scrivere una funzione che calcoli il prodotto scalare (\texttt{\Word{double}}) di due vettori 
   (di \texttt{\Word{double}}) :
  \end{itemize}

  \begin{equation}
    \langle  v, w \rangle = \sum_{i=1}^{N} v_i \cdot w_i
  \end{equation}

\end{frame}

\begin{frame}{Esercizi (III)}

  \begin{itemize}
    \item Scrivere un programma che data in ingresso una matrice di numeri casuali ne calcoli la matrice trasposta.
    \begin{itemize}
      \item Scrivere una funzione che stampi una matrice
      \item Scrivere una funzione che calcoli la matrice trasposta data una matrice
    \end{itemize}
  \end{itemize}

  Data la matrice $A$, la matrice trasposta - indicata con $A^T$ - si ottiene scambiando le righe con le colonne:
  \begin{columns}[T]
    \begin{column}[T]{5cm}
      \begin{equation*}
	A = \left(
	\begin{array}{ccc}
	  1 & 9 & 3 \\
	  0 & 4 & 5 \\
	  2 & 7 & 0 \\
	\end{array} \right)
      \end{equation*}      
    \end{column}
    \begin{column}[T]{5cm}
      \begin{equation*}
	  A^T = \left(
	  \begin{array}{ccc}
	    1 & 0 & 2 \\
	    9 & 4 & 7 \\
	    3 & 5 & 0 \\
	  \end{array} \right)
      \end{equation*}  
    \end{column}
  \end{columns}

\end{frame}

\begin{frame}{Esercizi (IV)}
  \begin{itemize}
    \item Scrivere un programma che dati in ingresso i tre coefficienti 
	  $a$, $b$ e $c$ risolva le equazioni di secondo grado usando la formula.
    \begin{itemize}
      \item il programma deve gestire i casi in cui non esistano soluzioni reali
      (discriminante $\Delta = (b^2 - 4ac)$ negativo),
    \end{itemize}

  \end{itemize}

  \begin{equation*}
    x_{1,2} = \dfrac{-b \pm \sqrt{b^2 - 4ac}}{2a}
  \end{equation*}

\end{frame}
