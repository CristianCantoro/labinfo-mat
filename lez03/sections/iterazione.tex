\begin{frame}{Iterazione e ricorsione (I)}

  Una distizione importante per quanto riguarda le funzione è la
  seguente:
  \begin{itemize}
   \item funzioni \textbf{iterative}
   \item funzioni \textbf{ricorsive}
  \end{itemize}
  
  Una funzione \alert{\textbf{ricorsiva}} è una funzione che
  \textbf{richiama se stessa}. Altrimenti è \structure{\textbf{iterativa}}.

  Le funzioni viste finora sono iterative.
\end{frame}

\begin{frame}{Iterazione e ricorsione (II)}

  Le funzioni ricorsive si trovano molto comunemente anche in matematica.
  
  Esempi:
  \begin{itemize}
   \item fattoriale: $n! = n \cdot n - 1 \cdot \dots \cdot 3 \cdot 2 \cdot 1 := n \cdot (n - 1)!$
   \item successione di Fibonacci: $F_{n} = F_{n-1} + F_{n-2}$
  \end{itemize}
  ovviamente la definzione deve avere alcuni \textbf{casi limite} altrimenti si andrebbe
  avanti all'infinito.

  ${}$\newline

  \begin{minipage}{0.8\paperwidth}
    Per i casi di cui sopra:
    \begin{itemize}
      \item fattoriale: $n = 0 \Rightarrow 0! = 1$ (il fattoriale non è definito per numeri interi negativi)
      \item successione di Fibonacci: $F_{0} = 0$ e $F_{1} = 1$\footnote{Tradizionalmente si definiva 
	    $F_{0} = 1$ e $F_{1} = 1$, ma ora si definisce di solito come sopra. Si veda anche \url{https://oeis.org/A000045}}
    \end{itemize}
  \end{minipage}

\end{frame}

\begin{frame}[fragile]\frametitle{Definizione ricorsiva del fattoriale}

  \begin{JavaCodePlain}[commandchars=\\!|]
  \Jpublic \Jclass FattorialeRicorsivo {

    \Jpublic \Jstatic \Jint fattoriale(\Jint n) {
      \Jif[n == 0] {
        \Jreturn 1;
      }
      \Jreturn n * fattoriale(n - 1);
    }

    \Jpublic \Jstatic \Jvoid main(String[] \Jargs) {
      \Jint numero = 10;
      \Jint risultato = 0;
      
      risultato = fattoriale(numero);
      \Jprintf["Il fattoriale di \%d è: \%d"][numero, risultato];
    }
  }
  \end{JavaCodePlain}
\end{frame}