% \begin{frame}{Array (I)}
% 
%   Supponiamo di volere avere delle variabili che contegano tutte
%   le lettere (minuscole) dell'alfabeto (\texttt{\Word{char}}).
% 
% 
%   Dovere gestire 26 variabili potrebbe essere complicato!
% 
% 
%   Possiamo usare un \textbf{array}. 
%   
% \end{frame}

  \begin{frame}{Array (I)}
  \begin{table}[]
  \centering
    \begin{tabular}{cccccccccc}
      \multicolumn{10}{c}{}                                                                                                                                                                                                                                                                                                                                                                                                                                                                     \\
      \rowcolor[HTML]{EFEFEF} 
      {\color[HTML]{FE0000} \textbf{0}}               & \textbf{1}                                     & \textbf{2}                                     & \textbf{3}                                     & \textbf{4}                                     & \textbf{5}                                     & \textbf{6}                                     & \textbf{7}                                     & \textbf{8}                                     & \textbf{9}                                     \\ \hline
      \rowcolor[HTML]{9AFF99} 
      \multicolumn{1}{|c|}{\cellcolor[HTML]{9AFF99}\String{"a"}} & \multicolumn{1}{c|}{\cellcolor[HTML]{9AFF99}\String{"b"}} & \multicolumn{1}{c|}{\cellcolor[HTML]{9AFF99}\String{"c"}} & \multicolumn{1}{c|}{\cellcolor[HTML]{9AFF99}\String{"d"}} & \multicolumn{1}{c|}{\cellcolor[HTML]{9AFF99}\String{"e"}} & \multicolumn{1}{c|}{\cellcolor[HTML]{9AFF99}\String{"f"}} & \multicolumn{1}{c|}{\cellcolor[HTML]{9AFF99}\String{"g"}} & \multicolumn{1}{c|}{\cellcolor[HTML]{9AFF99}\String{"h"}} & \multicolumn{1}{c|}{\cellcolor[HTML]{9AFF99}\String{"i"}} & \multicolumn{1}{c|}{\cellcolor[HTML]{9AFF99}\String{"j"}} \\ \hline
      \multicolumn{10}{c}{}                                                                                                                                                                                                                                                                                                                                                                                                                                                      
    \end{tabular}
  \end{table}  
  
  Un \textbf{array} (o \textbf{vettore}) è una \emph{struttura dati} composto da un \textbf{numero finito} di 
  elementi tutti dello \textbf{stesso tipo}.

\end{frame}

\begin{frame}{Array (I)}
  
  Simili al concetto matematico di vettore:
  
  \begin{equation*}
    v = \left( \begin{array}{c}
                1 \\
                0 \\
                4 \\
                7 \\
               \end{array}\right)
  \end{equation*}

\end{frame}

\begin{frame}{Array (II)}
  \begin{enumerate}
   \item Dichiarazione di un array:
  \end{enumerate}
  \begin{center}
    \begin{columns}[T]
      \begin{column}[T]{3cm}
	\begin{itemize}
	 \item 	\texttt{\Violet{tipo} \Green{nome}\textbf{[]}}
	\end{itemize}
      \end{column}
      \begin{column}[T]{3cm}
	\begin{itemize}
	 \item \texttt{\Violet{tipo}\textbf{[]} \Green{nome}}
	\end{itemize}
      \end{column}
    \end{columns}   
  \end{center}

  \begin{enumerate}
    \setcounter{enumi}{1}
    \item Allocazione:
  \end{enumerate}
  \begin{itemize}
   \item \texttt{\Green{nome} = \Red{\textbf{new}} \Violet{tipo}[\Blue{dimensione}]}
  \end{itemize}

  \begin{enumerate}
    \setcounter{enumi}{2}
    \item Diachiarazione e inizializzazione con literals:
  \end{enumerate}
  \begin{itemize}
   \item \texttt{\Violet{tipo}\textbf{[]} \Green{nome} =\textbf{\{}val1, val2, ...\textbf{\}}} 
  \end{itemize}

\end{frame}

\begin{frame}[fragile]\frametitle{Array (III)}

  \begin{JavaCodePlain}[commandchars=\\!|]
  \Jpublic \Jclass Array {

    \Jpublic \Jstatic \Jvoid main(String[] args) {
    
      \Jint[] vettore;      
      vettore = \Word!new| int[10];

    \dots
  \end{JavaCodePlain}

\end{frame}

\begin{frame}{Array (IV)}
  \begin{table}[]
    \centering
    \begin{tabular}{cccccccccc}
      \multicolumn{10}{c}{\textbf{indici}}                                                                                                                                                                                                                                                                                                                                                                                                                                                                     \\
      \rowcolor[HTML]{EFEFEF} 
      {\color[HTML]{FE0000} \textbf{0}}               & \textbf{1}                                     & \textbf{2}                                     & \textbf{3}                                     & \textbf{4}                                     & \textbf{5}                                     & \textbf{6}                                     & \textbf{7}                                     & \textbf{8}                                     & \textbf{9}                                     \\ \hline
      \rowcolor[HTML]{9AFF99} 
      \multicolumn{1}{|c|}{\cellcolor[HTML]{9AFF99}\String{"a"}} & \multicolumn{1}{c|}{\cellcolor[HTML]{9AFF99}\String{"b"}} & \multicolumn{1}{c|}{\cellcolor[HTML]{9AFF99}\String{"c"}} & \multicolumn{1}{c|}{\cellcolor[HTML]{9AFF99}\String{"d"}} & \multicolumn{1}{c|}{\cellcolor[HTML]{9AFF99}\String{"e"}} & \multicolumn{1}{c|}{\cellcolor[HTML]{9AFF99}\String{"f"}} & \multicolumn{1}{c|}{\cellcolor[HTML]{9AFF99}\String{"g"}} & \multicolumn{1}{c|}{\cellcolor[HTML]{9AFF99}\String{"h"}} & \multicolumn{1}{c|}{\cellcolor[HTML]{9AFF99}\String{"i"}} & \multicolumn{1}{c|}{\cellcolor[HTML]{9AFF99}\String{"j"}} \\ \hline
      \multicolumn{10}{c}{{\color[HTML]{009901} valori}}                                                                                                                                                                                                                                                                                                                                                                                                                                                      
    \end{tabular}
  \end{table}
  
  Per accedere all'elemento con indice \texttt{\Blue{i}} dell'array si usa la sintassi:
  \begin{center}
    \texttt{\Green{nome}\textbf{[}\Blue{i}\textbf{]}}
  \end{center}
  
  Notate che:
  \begin{itemize}
   \item gli indici partono da \alert{\textbf{zero}} e arrivono fino alla \structure{\textbf{lunghezza dell'array meno uno}}
   \item il primo elemento è accessibile con \texttt{\Green{nome}\textbf{[}\Blue{0}\textbf{]}}
   \item l'ultimo elemento è accessibile con \texttt{\Green{nome}\textbf{[}\Blue{len - 1}\textbf{]}} (dove \texttt{len}
	 è la lunghezza dell'array.
  \end{itemize}

\end{frame}

\begin{frame}{Array (V)}
  \begin{table}[]
    \centering
    \begin{tabular}{cccccccccc}
      \multicolumn{10}{c}{\textbf{indici}}                                                                                                                                                                                                                                                                                                                                                                                                                                                                     \\
      \rowcolor[HTML]{EFEFEF} 
      {\color[HTML]{FE0000} \textbf{0}}               & \textbf{1}                                     & \textbf{2}                                     & \textbf{3}                                     & \textbf{4}                                     & \textbf{5}                                     & \textbf{6}                                     & \textbf{7}                                     & \textbf{8}                                     & \textbf{9}                                     \\ \hline
      \rowcolor[HTML]{9AFF99} 
      \multicolumn{1}{|c|}{\cellcolor[HTML]{9AFF99}\String{"a"}} & \multicolumn{1}{c|}{\cellcolor[HTML]{9AFF99}\String{"b"}} & \multicolumn{1}{c|}{\cellcolor[HTML]{9AFF99}\String{"c"}} & \multicolumn{1}{c|}{\cellcolor[HTML]{9AFF99}\String{"d"}} & \multicolumn{1}{c|}{\cellcolor[HTML]{9AFF99}\String{"e"}} & \multicolumn{1}{c|}{\cellcolor[HTML]{9AFF99}\String{"f"}} & \multicolumn{1}{c|}{\cellcolor[HTML]{9AFF99}\String{"g"}} & \multicolumn{1}{c|}{\cellcolor[HTML]{9AFF99}\String{"h"}} & \multicolumn{1}{c|}{\cellcolor[HTML]{9AFF99}\String{"i"}} & \multicolumn{1}{c|}{\cellcolor[HTML]{9AFF99}\String{"j"}} \\ \hline
      \multicolumn{10}{c}{{\color[HTML]{009901} valori}}                                                                                                                                                                                                                                                                                                                                                                                                                                                      
    \end{tabular}
  \end{table}
  
  Per accedere all'elemento con indice \texttt{\Blue{i}} dell'array si usa la sintassi:
  \begin{center}
    \texttt{valore = \Green{nome}\textbf{[}\Blue{i}\textbf{]}}
  \end{center}
  
  Esempi:
  \begin{itemize}
   \item \texttt{\Green{nome}\textbf{[}\Blue{0}\textbf{]}} $\rightarrow$ \String{"a"}
   \item \texttt{\Green{nome}\textbf{[}\Blue{4}\textbf{]}} $\rightarrow$ \String{"e"}
   \item \texttt{\Green{nome}\textbf{[}\Blue{10}\textbf{]}} $\rightarrow$ \alert{\textbf{Errore: out of range}}
  \end{itemize}

\end{frame}

\begin{frame}{Array (VI)}
  \begin{table}[]
    \centering
    \begin{tabular}{cccccccccc}
      \multicolumn{10}{c}{\textbf{indici}}                                                                                                                                                                                                                                                                                                                                                                                                                                                                     \\
      \rowcolor[HTML]{EFEFEF} 
      {\color[HTML]{FE0000} \textbf{0}}               & \textbf{1}                                     & \textbf{2}                                     & \textbf{3}                                     & \textbf{4}                                     & \textbf{5}                                     & \textbf{6}                                     & \textbf{7}                                     & \textbf{8}                                     & \textbf{9}                                     \\ \hline
      \rowcolor[HTML]{9AFF99} 
      \multicolumn{1}{|c|}{\only<1-2>{\cellcolor[HTML]{9AFF99}\String{"a"}}\only<3->{\cellcolor[HTML]{34FF34}\String{"z"}}} & \multicolumn{1}{c|}{\cellcolor[HTML]{9AFF99}\String{"b"}} & \multicolumn{1}{c|}{\cellcolor[HTML]{9AFF99}\String{"c"}} & \multicolumn{1}{c|}{\cellcolor[HTML]{9AFF99}\String{"d"}} & \multicolumn{1}{|c|}{\only<-4>{\cellcolor[HTML]{9AFF99}\String{"e"}}\only<5->{\cellcolor[HTML]{34FF34}\String{"x"}}} & \multicolumn{1}{c|}{\cellcolor[HTML]{9AFF99}\String{"f"}} & \multicolumn{1}{c|}{\cellcolor[HTML]{9AFF99}\String{"g"}} & \multicolumn{1}{c|}{\cellcolor[HTML]{9AFF99}\String{"h"}} & \multicolumn{1}{c|}{\cellcolor[HTML]{9AFF99}\String{"i"}} & \multicolumn{1}{c|}{\cellcolor[HTML]{9AFF99}\String{"j"}} \\ \hline
      \multicolumn{10}{c}{{\color[HTML]{009901} valori}}                                                                                                                                                                                                                                                                                                                                                                                                                                                      
    \end{tabular}
  \end{table}
  
  Per impostare il valore dell'elemento con indice \texttt{\Blue{i}} dell'array si usa la sintassi:
  \begin{center}
    \texttt{\Green{nome}\textbf{[}\Blue{i}\textbf{]} = valore}
  \end{center}
  
  Esempi:
  \pause{
  \begin{itemize}
   \item<2-> \texttt{\Green{nome}\textbf{[}\Blue{0}\textbf{]} = \String{"z"}}
   \item<4-> \texttt{\Green{nome}\textbf{[}\Blue{4}\textbf{]} = \String{"x"}}
   \item<6-> \texttt{\Green{nome}\textbf{[}\Blue{10}\textbf{]}} $\rightarrow$ \alert{\textbf{Errore: out of range}}
  \end{itemize}
  }
\end{frame}

\begin{frame}[fragile]{Array e funzioni}

  Gli array possono essere usati come argomenti delle funzioni
  
  \begin{JavaCodePlain}[commandchars=\\!|]
  \Jpublic \Jclass ArrayFun {

    \Jpublic \Jstatic \Jvoid somma(int[] v) {
    
      int s = 0,
      \Jfor[int i = 0][i < v.length]!i++| {
	
      }
    \dots
  \end{JavaCodePlain}
  
\begin{alertblock}{Attenzione}
  A differenza delle altre variabili, se  un  array viene modificato in una funzione resta 
  modificato per tutto il programma
\end{alertblock}

\end{frame}
